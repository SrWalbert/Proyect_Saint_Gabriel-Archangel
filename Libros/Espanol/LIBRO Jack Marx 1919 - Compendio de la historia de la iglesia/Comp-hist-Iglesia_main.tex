\raggedbottom{} \documentclass[12pt]{book}
\usepackage[utf8]{inputenc} % Linux
\usepackage[T1]{fontenc}% Set the font (output) encodings
\usepackage{graphicx} % Required for inserting images
\usepackage[spanish]{babel}
\usepackage{hyphenat} % For other packages
\usepackage{adjustbox} % For all content (especialy matrix) get out the margin
\usepackage[left=2.54cm, right=2.54cm, top=2.54cm, bottom=2.54cm]{geometry} % For margin on general document. 2.54cm on every side is for Appa style. Free of change it.
% Especials begg. --------------------------------------------------------
% Here the packages you'll use just in this document

\usepackage{amsmath} % Matemetics symbol on matrix, vectors, etc. Also Bold font
\usepackage{amssymb} % Other mathematical symbols like R on real numbers.
%\usepackage{amsfonts} % Optional,for founts of american matematical society
%\usepackage{setspace} % Optional for interline spacing

\usepackage{xcolor} % For colours on document
% Color definition Begg. ----------------------

  \definecolor{light-blue}{rgb}{0.17, 0.40, 0.69} 

% Color definition End ------------------------

% Especials end ----------------------------------------------------------

% Temporal packages begg. ------------------------------------------------
% For temporal packages you'll delete on the final version

% Temporal packages end --------------------------------------------------

% User Data begg. ---------------------
\title{Compendio de historia de la Iglesia}
\author{Jackob Marx}
\date{1919}
% LIBRO Jack Marx 'der jüngere' 1919 - Compendio de la historia de la Iglesia - Librería Religiosa (de Sexta Edición original) Traducido por R.P. Ramón Ruiz Amado
% User Data end -----------------------

\begin{document}
% \maketitle % Optional for a fast titlepage
\begin{titlepage}
  \begin{center}
    {\Huge \textbf{Compendio de historia de la Iglesia} } \\
    \vspace{6mm}
    {\huge Jack Marx}\\
    \vspace{1mm}
    {Doctor en teología y filosofía}\\%18cm entre esto y traducido por
    \vspace{18cm}
    {\large Traducido por \textbf{R. P. Ramón Luís Armando}}
  \end{center}
\end{titlepage}
\tableofcontents
\chapter{Presentación del libro}
\section{Información general}
\begin{flushleft}
  \noindent UNIVERSITY OF CALIFORNIA, SAN DIEGO\\
  3 1822 02395 0058\\
  Compendio
  De
  Historia de la Iglesia\\
  Compuesto en Alemán por
  J. MARX, Doctor en Teologia y Filosofía\\
  Profesor de Hist.\ ecles.\ y Derecho canónico en el Seminario de Tréveris\\[2em]
  TRADUCIDO DE LA SEXTA EDICIÓN ORIGINAL
  POR EL
  R. P. RAMÓN RUIZ AMADO, S. J.\\
  ABRERIAL RELICIOSA\\
  BARCELONA\\
  LIBRERÍA RELIGIOSA, AVIÑÓ, 20\\
  MCMXIX\\[1em]
  \noindent LIBRERIA LEHMANN MUTER \& COM ENCUADERNACION SE C.R.\\[1em]
  \noindent UNIVERSITY OF CALIFORNIA, SAN DIEGO\\
  3 1822 02395 0058
  GEISEL LIBRARY\\
  UNIVERSITY OF CALIFORNIA, SAN DIEGO\\
  LA JOLLA, CALIFORNIA\\

  PROPERTY
  OF
  M.ARCE\\
  COMPENDIO
  DE
  Historia de la Iglesia\\
  COMPUESTO EN ALEMÁN POR
  J. MARX, Doctor en Teología y Filosofia
  Profesor de Hist.\ ecles.\ y Derecho canónico en el Seminario de Tréveris
  TRADUCIDO DE LA SEXTA EDICIÓN ORIGINAL
  POR EL
  R. P. RAMÓN RUIZ AMADO, S. J.
  IBRERIAN
  TRIA-PARLICIOSA
  BARCELONA
  LIBRERÍA RELIGIOSA, AVIÑÓ, 20
  MCMXIX
\end{flushleft}
\section{Sobre esta reedición digital}
Bienvenido estimado Lector. He aquí mi priemera obra reeditada, capturada de físico por un anónimo, salvada en formato de texto plano (.txt) de la www.z-library.sk mediante www.annas-archive.org\@. Este es un trabajo gratuito sobre una obra redactada hace más de cien años, con fin de preservar la cultura y bibliografía católica. Siéntete libre de compartirla (sin ánimo de lucro), siempre respetando el nombre del autor original.

Como parte del trabajo de reedición se ha revisado la ortografía del documento con base en las reglas de la Real Academia de la Lengua Española, a modo que se cumplan sus criterios sobre la correcta forma de expresar la escritura.
Encomendando este proyecto a San Gabriel Arcángel y San Francisco de Sales: Atentamente Walbert Isaac Trejo Ayala.
\section{Prefacio}
\begin{quotation}
  Enitendum magnopere, ut omnia ementita et falsa, adeundis rerum fontibus, refutentur; et illud in primis scribentium (historiam) obversetur animo, primam esse historiae legem, ne quid falsi dicere audeat, deinde ne quid veri non audeat; ne qua suspicio gratiae sit in scribendo, ne qua simultatis.\ -Est autem in scholarum usum confectio commentariorum necessaria, qui salva veritate et nullo adolescentium periculo ipsam artem historicam ilustrare et augere queant.\ Epist.\ Leonis PP.\@ XIII.\@ d.\@ 18.\@ Aug.\@ 1883.
\end{quotation}

\begin{quotation}
  Parce que l'église, qui continue parmi les hommes la vie du Verbe ipcarné, se compose d'un élément divin et d'un élément humain, ce dernier doit être exposé par les
  maîtres et étudié par les élèves avec une grande probité, comme il est dit au livre de
  Job: «Dieu n'a pas besoin de nos mensonges». L'historien de l'église sera d'autant plus
  fort pour faire ressortir son origine divine, supérieure à tout concept d'ordre purement
  terrestre et naturel, qu'il aura été loyal à ne rien dissimuler des épreuves que les fautes
  de ses enfants, et parfois même de ses ministres, ont fait subir à cette Epouse du Christ
  dans le cours des siècles.
  Encycl.\@ eiusdem d.\ 8.\ Sept.\ 1899.
\end{quotation}

\noindent NIHIL OBSTAT
El Censor,
RAMÓN LLOBEROLA
Barcelona, 22 de Diciembre de 1913
IMPRIMATUR
IMPRIMI POTEST
JOSEPHUS BARRACHINA, S. J.
Praepositus provinciae Aragoniae
El Vicario Capitular,
JOSÉ PALMAROLA
Por mandato de Su Sría.,
LIC.\@ SALVADOR CARRERAS, PBRO.,
Scrio. Canc.
\quad::\, Reservados\,:: \quad
todos los derechos.
\section{Aprovaciones pontificias}
\subsection{Primera carta}
\begin{center}
  \large SEGRETARIA DI STATO DI SUA SANTITÁ DAL VATICANO
\end{center}
\begin{flushright}
  13 de Junio de 1914\\
  Reverendísimo Señor Profesor:
\end{flushright}

El Santo Padre me encarga hacer llegar a V.\ sus augustas acciones de gracias, por el homenaje que le ha hecho V. poco ha de su MANUAL DE HISTORIA ECLESIÁSTICA, traducido al italiano por el Rev.\, Sac.\, Doctor G.\,B.\, Pagnini. Su Santidad se ha alegrado mucho de saber la copiosa difusión de la obra susodicha, especialmente en los Seminarios, y presume que la misma habrá contribuído no poco, y contribuirá en lo sucesivo, a proveer a los jóvenes clérigos de aquella sana cultura histórica, que es tan útil, y aun necesaria, para la plena inteligencia de la Doctrina de la Iglesia y para la defensa de la verdad.

En prenda de su paternal benevolencia, el augusto Pontífice envía a V.\ cordialmente, su Apostólica Bendición.\  De muy buena voluntad le añado mis personales acciones de gracias, por el ejemplar de dicha obra que cortesmente me dedicó, y aprovecho la ocasión presente para repetirme con sinceros sentimientos de estimación.

De V.\ S.\ Reverendísima affmo.\ servidor
R.\, CARD.\, MERRY DEL VAL
Revmo. Sac. Dr.\, J.\,Marx Prof.\ en el Seminario de Tréveris.
N.° 71855
\subsection{Segunda carta}
\begin{quotation}
  Reverende Pater:
  DAL VATICANO DIE 17 Junii 1914
  Et oblatum a te volumen HISTORIAM ECCLESIASTICAM Doctoris
  Marx in hispanicam linguam versam complectens, et addictissimae voluntatis sensus, qua illud offerebas, pergrata Beatissimo
  Patri fuisse scito.
  Adrite formandum iuniorem clerum eumdemque comparandum
  sacro ministerio digne fructuoseque fungendo, mirum quantum
  libri valent, qui inoffenso decurri possint pede, et magistri quorum labia custodiendae scientiae sunt assueta! Horum te in numero versari: hac florere laude redditum a te hispanice volumen,
  multorum consensu exploratum est: spemque id optimam facit
  susceptos a te, tam pio consilio, labores fore discentibus perutiles, scientibusque minime iniucundos.
  Hisce votis Sanctitas Sua, de pietatis officio gratias agens,
  tibi ex animo benedicit caelestiaque precatur munera.
  Hac eadem mente tibi gratulor et gratias ipse ago pro volumine mihi perhumaniter destinato, meque Paternitati tuae profiteor
  Addictissimum
  R.\ CARD.\ MERRY DEL VAL
  Reverendo Domino P. Raymundo Ruiz Amado, S. I. in Collegio
  S. Ignatii.-Barcinonem.
\end{quotation}
Del Vaticano, día 17 de Junio de 1914.
Reverendo Padre:
Participo a V. que han sido gratísimos al Santo Padre, el tomo
por V. ofrecido de la HISTORIA ECLESIÁSTICA del Dr.\ Marx, traducido
a la lengua española, y los sentimientos de adictísima voluntad con
que se lo ofrecía.
Para formar debidamente al Clero joven, y prepararle a ejercitar
digna y fructuosamente el sagrado ministerio, es admirable cuánto sirven los libros que se pueden recorrer sin tropiezo, y los maestros cuyos labios están acostumbrados a custodiar la ciencia. Que es V. del
número de estos maestros; y que merece esta alabanza el libro por
V. traducido al castellano; cosa es averiguada por el concorde testimonio de muchos; y nos da las mejores esperanzas de que los trabajos
por V. emprendidos con tan religioso intento, serán por extremo útiles
a los estudiantes, y no dejarán de agradar a los doctos.
Deseando que así sea, Su Santidad, al dar a V. las gracias por su
filial obsequio, bendice a V. cordialmente y ruega al Señor le conceda
sus celestiales dones.
Con este mismo ánimo felicito a V. y le doy las gracias por el
tomo que con tanta cortesía me ha dedicado, y me profeso de su
Paternidad adictísimo
R.\ CARD.\ MERRY DEL VAL
Reverendo Señor P. Ramón Ruiz Amado, S. I. en el Colegio de
S. Ignacio.-Barcelona.
\section{De los prólogos del autor}
El presente Compendio de Historia de la Iglesia sirvió durante diez
años como texto para las explicaciones del Autor, impreso como manuscrito en una corta tirada, antes de darlo definitivamente a la publicidad. Sólo después de este largo tiempo de prueba se ha entregado al
comercio de libros, con el fin principal de poder ofrecerlo a los alumnos a un precio más reducido.
La primera de las cualidades que debe tener un buen libro de texto
es la claridad de estilo, y la perspicuidad en la disposición y agrupación de los hechos. Esta cualidad se ha procurado, no sólo moderando el número de los párrafos, sino distinguiendo en cada uno de
ellos, por la numeración y diversidad de tipos, la síntesis de toda la
materia, de la más detenida explanación de ella.
La segunda cualidad imprescindible es el exámen crítico imparcial
de los hechos, en el cual el Autor ha procurado ser escrupuloso aun
en los casos en que podía resultar algo poco edificante. Así lo exige,
no sólo la primera ley de la Historia: la veracidad, sino aun el criterio
católico; pues estamos profundamente convencidos de que la sincera
exposición de los defectos reales, aun de los que se han hallado en los
más elevados representantes de la Iglesia, no hace sino abrillantar
más su divino esplendor y grandeza.\\
\ldots En la segunda edición se ha dado lugar a la Patrología, para
satisfacer a los que no le dedican estudio especial como asignatura separada, y se ha refundido y ampliado el estudio de la Historia de las
Misiones\ldots
\ldots El Autor supone que los señores profesores explanan particularmente la Historia Eclesiástica de su propio país; lo cual no es posible
hacerlo en un libro de texto destinado para usarse en muy diversas
regiones\ldots
En las ediciones 5.a y 6.a se han ampliado los capítulos referentes
a la Historia de la Constitución de la Iglesia y de sus luchas en el terreno científico. Asimismo se ha incluído la noticia de los más recientes acaecimientos.
Sólo ahora, considerando ya su obra del todo desenvuelta, ha comenzado el Autor a dar licencia para que se traduzca a lenguas extranjeras, como ya se ha hecho al italiano y se está haciendo al inglés.
¡Plega a Dios bendecir esta obra, para que produzca abundantes
bienes y ofrezca un eficaz auxilio a muchos estudiantes de sagrada
Teología!
Tréveris, 22 de Octubre de 1912.
\section{Abreviaturas}
\begin{itemize}
  \item AAS.\@ Acta apostolicae sedis. Commentarium officiale, Romae 1909 ss.
  \item AA.\ SS.\@ Acta sanctorum, quotquot toto orbe coluntur, ed. Bollandus et alii, Antverp. 1643 sqq., 1/?
  \item AB.\@ Analecta Bollandiana, ed, de Smedt-van Hooff-de Backer, Paris-Bruxelles 1882, 1/?
  \item ASS.\@ Acta s.\@ sedis, Romae 1881 ss. 1/40.
  \item BR.\@ Magnum Bullarium Romanum a b. Leone magno usque ad Benedictum XIII.\ ed. Cherubini, Luxenburgii 1727, 1/17 f. BRC.\ Bullarii Romani continuatio, ed. Barberi-Speccia-Secreti, Romae 1835 sqq., 1/20 f.
  \item CG.\@ Hefele-Knöpfler, Konziliengeschichte, 2. A. Freib. 1873 ff., 1/9.
  \item CL.\@ Acta et decreta s.\ conciliorum recentiorum. Collectio Lacensis, Frib. 1870/86, 1/7.
  \item CSEL.\@ Corpus scriptorum ecclesiae latinorum, ed. Vindobonae 1866 sqq. 1/?
  \item JL.\@ Regesta Pontificum Romanorum ab condita ecclesia ad a. 1198, ed. 2a curantibus Kaltenbrunner-Ewald-Löwenfeld, Lips. 1885/7, 1/2.
  \item LP.\@ Le Liber Pontificalis, Texte, introduction et commentaire par L. Duchesne Paris 1886, 1/2.
  \item MG.\@ AA.\@ Monumenta Germaniae historica, ed. Pertz-Waitz-Dümmler, Hannov.-Berol. 1826, sqq. Auctores antiquissimi.
  \item EE.\@ Epistolae.
  \item LL.\@ Leges.
  \item SS.\@ Scriptores.
  \item MGP.\@ Monumenta Germaniae paedagogica ed. Kehrbach, Berol. 1886 sqq.
  \item Mgr.\@ Monografía.

  \item PG.\@ Migne, Patrologiae cursus completus, Patrologia graeca usque ad saec, XV., Par. 1857 sqq., 1/161.

  \item PL.\@ Migne, Patrol.\ cursus completus, Patrologia latina ab aevo apostolico usque ad Innocentium III., Par. 1854 sqq. 1/221.
  \item RHE.\@ Revue d'histoire ecclésiastique, Louvain 1900 ss.
  \item RQH.\@ Revue des questions historiques, Paris 1866 ss.
\end{itemize}
\chapter{Introducción}
\section{Concepto de Historia Eclesiástica}
\section{División de Historia Eclesiástica}
\section{Fuentes y ciencias auxiliares de la Historia Eclesiástica}
\section{Método de la Historia Eclesiástica}
\section{Desenvolvimiento de la Historia Eclesiástica}
\part{Época Pimera. Primer Período: de las persecuciones del Cristianismo hata el edicto de Constantino}
\chapter{Capítulo Primero: Estado religioso y social del mundo, a la aparición del Cristianismo}
\section{Estado del Judaísmo}
\section{Estado del paganismo en el Imperio Romano}
\subsection{Preparación de los paganos para el Cristianismo}
\subsection{Obstáculos para la propagación del Cristianismo}
\section{JesuCristo, Fundador de la Iglesia}
\section{La primitiva comunidad de Jerusalem}
\section{La Iglesia se separa de la Sinagoga. Admisión de los Gentiles}
\section{El apostol de las gentes: Pablo}
\section{San Pedro. Fundación de la Iglesia Romana}
\section{Los demás apóstoles. Discipulos de los apostoles}
\section{Causas y carácter de las persecuciones}
\section{Las persecuciones}
\section{Significado del Martirio para la Iglesia}
\section{El Cristianismo atacado con armasa intelectuales}
\section{Extensión del Cristianismo al fin de las persecuciones}
\chapter{Capítulo Segundo: Desarrollo de la Doctrina Eclesiástica. Herejías}
\section{Herejes judaizantes}
\section{Gnosticismo}
\section{Los monarquianos}
\section{Ciencia y literatura cristiana}
\section{Desarrollo de la Doctrina Eclesiástica}
\chapter{Capítulo Tercero: Constitución de la Iglesia. Culto y Disciplina: rasgos fundamentales de la constitución eclesiástica}
\section{Desarrollo de la constitución}
\section{Vida del clero}
\section{Primado del Obispo de Roma}
\section{Bautismo y Confirmación. Controversia sobre el bautismo de los herejes}
\section{La disciplina penitencial. Cismas}
\section{El culto. Las fiestas. Controversia sobre la Pascua}
\section{La vida cristiana}
\part{Época Primera. Segundo Período: Época de luchas Dogmáticas}
\chapter{Capítulo Primero: historia externa de la Iglesia}
\section{Cristianismo fuera del Imperio Romano}
\section{Ruina del paganismo en el Imperio Romano}
\section{La Iglesia y el Imperio Romano}
\chapter{Capítulo Segundo: Desenvolvimiento científico. Herejías y concilios}
\section{Los donatistas}
\section{El arrianismo. Primer Consilio Universal}
\section{Derivaciones del Arrianismo y herejías simultáneas}
\section{El priscilianismo. Sectas menores}
\section{El pelagianismo}
\section{El nestorianismo}
\section{El monofisitismo}
\section{El monotelismo}
\section{Ciencia y literatura Eclesiástica}
\chapter{Capítulo Tercero: La constitución Eclesiástica}
\section{Perfeccionamiento y multiplicación de los oficios eclesiásticos}
\section{Primado del Obispo de Roma}
\section{Los Sínodos}
\section{El Clero}
\section{Vida monástica}
\chapter{Capítulo Cuarto: Culto, disciplina y vida cristiana}
\section{Sacramentos: La Santa Misa}
\section{Derecho penal. Penitencia pública}
\section{Templos, ayunos, festividades religiosas}
\section{Culto de los Santos y sus reliquias. Romerías}
\section{La vida cristiana}
\part{Época Segunda. Tercer Período: Desde la irrupción de los bárbaros hasta el pontificado de San Gregorio VII}
\chapter{Capítulo Primero: Extensión y limitación de la Iglesia}
\section{Las incursiones de los bárbaros y sus próximos efectos}
\section{Conversión de los Francos}
\section{El Cristianismo en las islas Británicas}
\section{Conversión de Alemania}
\section{Conversión de los pueblos Escandinavos}
\section{Conversión de los Eslavos y los Húngaros}
\section{El Islam como enemigo de la Cristiandad}
\chapter{Capítulo Segundo: El Pontificado y el Imperio. Estado y la Iglesia}
\section{Formación del Estado de la Iglesia}
\section{Restablecimiento del Imperio de Occidente}
\section{El pontificado y el Imperio desde San CarloMagno hasta San Gregorio VII}
\section{El Estado y la Iglesia en los reinos germánicos}
\chapter{Capítulo Tercero: Desarrollo de la Doctrina. Herejía y Cismas}
\section{Los iconoclastas y el VII Consilio Universal}
\section{El Cisma Griego. El VIII Consilio Universal}
\section{Controversias dogmáticas en Occidente}
\section{La Ciencia cristiana}
\chapter{Capítulo Cuarto: Constitución, disciplina, culto, vida cristiana}
\section{La Jerarquía. Erección de diócesis}
\section{Colecciones de Cánones. El Seudo-Isidoro}
\section{El Clero. La vida monástica}
\section{El culto. Veneración de los Santos}
\section{Disciplina y vida cristiana}
\part{Período Cuarto: Florecimiento de la Iglesia en la Edad Media (1073~1307)}
\chapter{Capítulo Primero: El Pontificado y el Imperio. El Estado y la Iglesia}
\section{Las ideas gregorianas}
\section{Conatos de reforma anteriores a San Gregorio VII}
\section{La contienda de las investiduras. San Gregorio VII}
\section{Consecuencias de la contiendad de las investiduras. X Consilio Universal}
\section{Lucha de los Emperadores de la Casa de Suabia con los papas. Inocencio III}
\section{Contiendas eclesiásticas en Inglaterra}
\section{El Pontificado cae bajo la influencia francesa}
\chapter{Capítulo Segundo: Historia externa de  la Iglesia}
\section{La lucha contra el Islamismo en Europa}
\section{Las Cruzadas}
\section{Lucha contra el Paganismo. Las Misiones}
\section{Los judíos en la edad media}
\chapter{Capítulo Tercero: Desarrollo de la vida monástica}
\section{Nuevas Órdenes con reglas antiguas}
\section{Las dos grandes Órdenes Mendicantes}
\section{Las Órdenes Militares}
\chapter{Capítulo Cuarto: La Doctrina de la Iglesia y sus adversarios}
\section{Impugnaciones de la Iglesia. Las sectas}
\section{La Inquisición}
\section{La Ciencia eclesiástica. Escolástica y Mística}
\chapter{Capítulo Quinto: Constitución, culto, vida cristiana}
\section{Desarrollo de la Constitución Eclesiástica}
\section{El culto. Los Sacramentos}
\section{El Arte cristiano}
\section{Vida y disciplina}
\part{Periodo Quinto: Obscurecimiento de la Autoridad Papal. Decadencia del Imperio}
\chapter{Capítulo Primero: Historia del Pontificado}
\section{El destierro de Aviñón}
\section{Lucha del Papa con Luis de Baviera}
\section{Consecuencias del Cisma. Consilio de Basiela}
\section{Cisma de Occidente. Consilio de Constanza}
\section{Los papas de la época del renacimiento}
\chapter{Capítulo Segundo. Desenvolvimiento de la Doctrina. Herejías. Ciencia Eclesiástica}
\section{}

\end{document}