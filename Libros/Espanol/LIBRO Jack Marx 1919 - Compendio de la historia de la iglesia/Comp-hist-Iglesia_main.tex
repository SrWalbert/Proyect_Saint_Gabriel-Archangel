\documentclass[12pt]{book}
\usepackage[utf8]{inputenc} % Linux
\usepackage[T1]{fontenc}% Set the font (output) encodings
\usepackage{graphicx} % Required for inserting images
\usepackage[spanish]{babel}
\usepackage{hyphenat} % For other packages
\usepackage{adjustbox} % For all content (especialy matrix) get out the margin
\usepackage[left=2.54cm, right=2.54cm, top=2.54cm, bottom=2.54cm]{geometry} % For margin on general document. 2.54cm on every side is for Appa style. Free of change it.
% Especials begg. --------------------------------------------------------
% Here the packages you'll use just in this document

\usepackage{amsmath} % Matemetics symbol on matrix, vectors, etc. Also Bold font
\usepackage{amssymb} % Other mathematical symbols like R on real numbers.
%\usepackage{amsfonts} % Optional,for founts of american matematical society
%\usepackage{setspace} % Optional for interline spacing

\usepackage{xcolor} % For colours on document
% Color definition Begg. ----------------------

  \definecolor{light-blue}{rgb}{0.17, 0.40, 0.69} 

% Color definition End ------------------------

% Especials end ----------------------------------------------------------

% Temporal packages begg. ------------------------------------------------
% For temporal packages you'll delete on the final version

% Temporal packages end --------------------------------------------------

% User Data begg. ---------------------
\title{Compendio de historia de la Iglesia}
\author{Jackob Marx}
\date{1919}
% LIBRO Jack Marx 'der jüngere' 1919 - Compendio de la historia de la Iglesia - Librería Religiosa (de Sexta Edición original) Traducido por R.P. Ramón Ruiz Amado
% User Data end -----------------------

\begin{document}
% \maketitle % Optional for a fast titlepage
\begin{titlepage}
  \begin{center}
    {\Huge \textbf{Compendio de historia de la Iglesia} } \\
    \vspace{6mm}
    {\huge Jack Marx}\\
    \vspace{1mm}
    {Doctor en teología y filosofía}\\%18cm entre esto y traducido por
    \vspace{18cm}
    {\large Traducido por \textbf{R. P. Ramón Luís Armando}}

  \end{center}
\end{titlepage}
\tableofcontents % The table of contents
\chapter{Presentación del libro}
\section{Información general}
\noindent UNIVERSITY OF CALIFORNIA, SAN DIEGO
3 1822 02395 0058\\
Compendio
De
Historia de la Iglesia\\
Compuesto en Alemán por
J. MARX, Doctor en Teologia y Filosofía\\
Profesor de Hist.\ ecles.\ y Derecho canónico en el Seminario de Tréveris\\\\
TRADUCIDO DE LA SEXTA EDICIÓN ORIGINAL
POR EL
R. P. RAMÓN RUIZ AMADO, S. J.\\
ABRERIAL RELICIOSA\\
BARCELONA\\
LIBRERÍA RELIGIOSA, AVIÑÓ, 20\\
MCMXIX\\[1em]
\noindent LIBRERIA LEHMANN MUTER \& COM ENCUADERNACION SE C.R.\\[1em]
\noindent UNIVERSITY OF CALIFORNIA, SAN DIEGO\\
3 1822 02395 0058
GEISEL LIBRARY\\
UNIVERSITY OF CALIFORNIA, SAN DIEGO\\
LA JOLLA, CALIFORNIA\\

PROPERTY
OF
M.ARCE\\
COMPENDIO
DE
Historia de la Iglesia\\
COMPUESTO EN ALEMÁN POR
J. MARX, Doctor en Teología y Filosofia
Profesor de Hist.\ ecles.\ y Derecho canónico en el Seminario de Tréveris
TRADUCIDO DE LA SEXTA EDICIÓN ORIGINAL
POR EL
R. P. RAMÓN RUIZ AMADO, S. J.
IBRERIAN
TRIA-PARLICIOSA
BARCELONA
LIBRERÍA RELIGIOSA, AVIÑÓ, 20
MCMXIX

\section{Prefacio}
Enitendum magnopere, ut omnia ementita et falsa, adeundis rerum fontibus, refutentur; et illud in primis scribentium (historiam) obversetur animo, primam esse historiae legem, ne quid falsi dicere audeat, deinde ne quid veri non audeat; ne qua suspicio gratiae sit in scribendo, ne qua simultatis.-Est autem in scholarum usum confectio commentariorum necessaria, qui salva veritate et nullo adolescentium periculo ipsam artem historicam ilustrare et augere queant. Epist. Leonis PP.\@ XIII.\@ d.\@ 18.\@ Aug.\@ 1883.\\

Parce que l'église, qui continue parmi les hommes la vie du Verbe ipcarné, se com-
pose d'un élément divin et d'un élément humain, ce dernier doit être exposé par les
maîtres et étudié par les élèves avec une grande probité, comme il est dit au livre de
Job: «Dieu n'a pas besoin de nos mensonges». L'historien de l'église sera d'autant plus
fort pour faire ressortir son origine divine, supérieure à tout concept d'ordre purement
terrestre et naturel, qu'il aura été loyal à ne rien dissimuler des épreuves que les fautes
de ses enfants, et parfois même de ses ministres, ont fait subir à cette Epouse du Christ
dans le cours des siècles.
Encycl.\@ eiusdem d. 8. Sept. 1899.
NIHIL OBSTAT
El Censor,
RAMÓN LLOBEROLA
Barcelona, 22 de Diciembre de 1913
IMPRIMATUR
IMPRIMI POTEST
JOSEPHUS BARRACHINA, S. J.
Praepositus provinciae Aragoniae
El Vicario Capitular,
JOSÉ PALMAROLA
Por mandato de Su Sría.,
LIC.\@ SALVADOR CARRERAS, PBRO.,
Scrio. Canc.
\quad::\, Reservados\,:: \quad
todos los derechos

\end{document}