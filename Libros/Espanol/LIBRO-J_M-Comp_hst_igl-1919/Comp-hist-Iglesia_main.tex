\raggedbottom{} \documentclass[12pt, a4paper]{book}
\usepackage[utf8]{inputenc} % Linux
\usepackage[T1]{fontenc}% Set the font (output) encodings
\usepackage{graphicx} % Required for inserting images
\usepackage[spanish]{babel}
\usepackage{hyphenat} % For other packages
\usepackage{adjustbox} % For all content (especialy matrix) get out the margin
\usepackage[left=2.54cm, right=2.54cm, top=2.54cm, bottom=2.54cm]{geometry} % For margin on general document. 2.54cm on every side is for Appa style. Free of change it.
\usepackage[backend=biber,style=verbose-note]{biblatex}
\addbibresource{biblio.bib}

% Especials begg. --------------------------------------------------------
% Here the packages you'll use just in this document

\usepackage{amsmath} % Matemetics symbol on matrix, vectors, etc. Also Bold font
\usepackage{amssymb} % Other mathematical symbols like R on real numbers.
%\usepackage{amsfonts} % Optional,for founts of american matematical society
%\usepackage{setspace} % Optional for interline spacing
\usepackage{csquotes}
\usepackage{xcolor} % For colours on document
% Color definition Begg. ----------------------

  \definecolor{light-blue}{rgb}{0.17, 0.40, 0.69} 

% Color definition End ------------------------
\usepackage[colorlinks= true, citecolor=green, ]{hyperref}
% Especials end ----------------------------------------------------------

% Temporal packages begg. ------------------------------------------------
% For temporal packages you'll delete on the final version

% Temporal packages end --------------------------------------------------

% User Data begg. ---------------------
\title{Compendio de historia de la Iglesia}
\author{Jackob Marx}
\date{1919}
% LIBRO Jack Marx 'der jüngere' 1919 - Compendio de la historia de la Iglesia - Librería Religiosa (de Sexta Edición original) Traducido por R.P. Ramón Ruiz Amado
% User Data end -----------------------

\begin{document}
% \maketitle % Optional for a fast titlepage
\begin{titlepage}
  \begin{center}
    {\Huge \textbf{Compendio de historia de la Iglesia} } \\
    \vspace{6mm}
    {\huge Jack Marx}\\
    \vspace{1mm}
    {Doctor en teología y filosofía}\\%18cm entre esto y traducido por
    \vspace{18cm}
    {\large Traducido por \textbf{R. P. Ramón Luís Armando}}
  \end{center}
\end{titlepage}
{\hypersetup{linkcolor=black}
\tableofcontents
}
\chapter{Presentación del libro}
\section{Información general}
\begin{flushleft}
  \noindent UNIVERSITY OF CALIFORNIA, SAN DIEGO\\
  3 1822 02395 0058\\
  Compendio
  De
  Historia de la Iglesia\\
  Compuesto en Alemán por
  J. MARX, Doctor en Teologia y Filosofía\\
  Profesor de Hist.\ ecles.\ y Derecho canónico en el Seminario de Tréveris\\[2em]
  TRADUCIDO DE LA SEXTA EDICIÓN ORIGINAL
  POR EL
  R. P. RAMÓN RUIZ AMADO, S. J.\\
  ABRERIAL RELICIOSA\\
  BARCELONA\\
  LIBRERÍA RELIGIOSA, AVIÑÓ, 20\\
  MCMXIX\\[1em]
  \noindent LIBRERIA LEHMANN MUTER \& COM ENCUADERNACION SE C.R.\\[1em]
  \noindent UNIVERSITY OF CALIFORNIA, SAN DIEGO\\
  3 1822 02395 0058
  GEISEL LIBRARY\\
  UNIVERSITY OF CALIFORNIA, SAN DIEGO\\
  LA JOLLA, CALIFORNIA\\

  PROPERTY
  OF
  M.ARCE\\
  COMPENDIO
  DE
  Historia de la Iglesia\\
  COMPUESTO EN ALEMÁN POR
  J. MARX, Doctor en Teología y Filosofia
  Profesor de Hist.\ ecles.\ y Derecho canónico en el Seminario de Tréveris
  TRADUCIDO DE LA SEXTA EDICIÓN ORIGINAL
  POR EL
  R. P. RAMÓN RUIZ AMADO, S. J.
  IBRERIAN
  TRIA-PARLICIOSA
  BARCELONA
  LIBRERÍA RELIGIOSA, AVIÑÓ, 20
  MCMXIX
\end{flushleft}
\newpage
\section{Sobre esta reedición digital} %Por terminar
Bienvenido estimado Lector. He aquí mi priemera obra reeditada, capturada de físico por un anónimo, salvada en formato de texto plano (.txt) de la www.z-library.sk mediante www.annas-archive.org\@. Este es un trabajo gratuito sobre una obra redactada hace más de cien años, con fin de preservar la cultura y bibliografía católica. Siéntete libre de compartirla (sin ánimo de lucro), siempre respetando el nombre del autor original.

Como parte del trabajo de reedición se ha revisado la ortografía del documento con base en las reglas de la Real Academia de la Lengua Española, a modo que se cumplan sus criterios sobre la correcta forma de expresar la escritura, se añadieron ciertos comentarios en párrafos donde se considere necesario, pues este libro tiene un siglo de antiguedad; se formalizó la estructura de citas, para que queden en formato pie de página como era originalmente. Lamentablemente no todos lo parráfos están completos.\\
Encomendando este proyecto a San Gabriel Arcángel y San Francisco de Sales: Atentamente Walbert Isaac Trejo Ayala.

Obra en dominio público (cc0)
\section{Prefacio}
\begin{quotation}
  Enitendum magnopere, ut omnia ementita et falsa, adeundis rerum fontibus, refutentur; et illud in primis scribentium (historiam) obversetur animo, primam esse historiae legem, ne quid falsi dicere audeat, deinde ne quid veri non audeat; ne qua suspicio gratiae sit in scribendo, ne qua simultatis.\ -Est autem in scholarum usum confectio commentariorum necessaria, qui salva veritate et nullo adolescentium periculo ipsam artem historicam ilustrare et augere queant.\ Epist.\ Leonis PP.\@ XIII.\@ d.\@ 18.\@ Aug.\@ 1883.
\end{quotation}

\begin{quotation}
  Parce que l'église, qui continue parmi les hommes la vie du Verbe ipcarné, se compose d'un élément divin et d'un élément humain, ce dernier doit être exposé par les maîtres et étudié par les élèves avec une grande probité, comme il est dit au livre de Job: «Dieu n'a pas besoin de nos mensonges». L'historien de l'église sera d'autant plus fort pour faire ressortir son origine divine, supérieure à tout concept d'ordre purement terrestre et naturel, qu'il aura été loyal à ne rien dissimuler des épreuves que les fautes de ses enfants, et parfois même de ses ministres, ont fait subir à cette Epouse du Christ dans le cours des siècles.\\
  Encycl.\@ eiusdem d.\ 8.\ Sept.\ 1899.
\end{quotation}
\textit{
  \noindent NIHIL OBSTAT El Censor, RAMÓN LLOBEROLA Barcelona, 22 de Diciembre de 1913 IMPRIMATUR IMPRIMI POTEST JOSEPHUS BARRACHINA, S. J. Praepositus provinciae Aragoniae El Vicario Capitular, JOSÉ PALMAROLA Por mandato de Su Sría., LIC.\@ SALVADOR CARRERAS, PBRO., Scrio. Canc. \quad::\, Reservados\,:: \quad todos los derechos.
}
\newpage
\section{Aprobaciones pontificias}
\subsection{Primera carta}
\begin{center}
  \large SEGRETARIA DI STATO DI SUA SANTITÁ DAL VATICANO
\end{center}
\begin{flushright}
  13 de Junio de 1914\\
  Reverendísimo Señor Profesor:
\end{flushright}

El Santo Padre me encarga hacer llegar a V.\ sus augustas acciones de gracias, por el homenaje que le ha hecho V. poco ha de su MANUAL DE HISTORIA ECLESIÁSTICA, traducido al italiano por el Rev.\, Sac.\, Doctor G.\,B.\, Pagnini. Su Santidad se ha alegrado mucho de saber la copiosa difusión de la obra susodicha, especialmente en los Seminarios, y presume que la misma habrá contribuído no poco, y contribuirá en lo sucesivo, a proveer a los jóvenes clérigos de aquella sana cultura histórica, que es tan útil, y aún necesaria, para la plena inteligencia de la Doctrina de la Iglesia y para la defensa de la verdad.

En prenda de su paternal benevolencia, el augusto Pontífice envía a V.\ cordialmente, su Apostólica Bendición.\  De muy buena voluntad le añado mis personales acciones de gracias, por el ejemplar de dicha obra que cortesmente me dedicó, y aprovecho la ocasión presente para repetirme con sinceros sentimientos de estimación.

De V.\ S.\ Reverendísima affmo.\ servidor
R.\, CARD.\, MERRY DEL VAL
Revmo. Sac. Dr.\, J.\,Marx Prof.\ en el Seminario de Tréveris.
N.° 71855
\subsection{Segunda carta}
\begin{quotation}
  \begin{flushright}
    \textbf{Reverende Pater}:\\
    DAL VATICANO DIE 17 Junii 1914
  \end{flushright}

  \noindent Et oblatum a te volumen \textit{HISTORIAM ECCLESIASTICAM} Doctoris Marx in hispanicam linguam versam complectens, et addictissimae voluntatis sensus, qua illud offerebas, pergrata Beatissimo Patri fuisse scito. Adrite formandum iuniorem clerum eumdemque comparandum sacro ministerio digne fructuoseque fungendo, mirum quantum libri valent, qui inoffenso decurri possint pede, et magistri quorum labia custodiendae scientiae sunt assueta! Horum te in numero versari: hac florere laude redditum a te hispanice volumen, multorum consensu exploratum est: spemque id optimam facit susceptos a te, tam pio consilio, labores fore discentibus perutiles, scientibusque minime iniucundos.

  Hisce votis Sanctitas Sua, de pietatis officio gratias agens, tibi ex animo benedicit caelestiaque precatur munera. Hac eadem mente tibi gratulor et gratias ipse ago pro volumine mihi perhumaniter destinato, meque Paternitati tuae profiteor Addictissimum:\\
  R.\ CARD.\ MERRY DEL VAL
  Reverendo Domino P. Raymundo Ruiz Amado, S. I. in Collegio
  S. Ignatii.-Barcinonem.
\end{quotation}
\newpage
\begin{flushright}
  Del Vaticano, día 17 de Junio de 1914.\\
  \textbf{Reverendo Padre}:
\end{flushright}

Participo a V. que han sido gratísimos al Santo Padre, el tomo por V. ofrecido de la \textit{HISTORIA ECLESIÁSTICA} del Dr.\ Marx, traducido a la lengua española, y los sentimientos de adictísima voluntad con que se lo ofrecía.\\
Para formar debidamente al Clero joven, y prepararle a ejercitar digna y fructuosamente el sagrado ministerio, es admirable cuánto sirven los libros que se pueden recorrer sin tropiezo, y los maestros cuyos labios están acostumbrados a custodiar la ciencia. Que es V. del número de estos maestros; y que merece esta alabanza el libro por V. traducido al castellano; cosa es averiguada por el concorde testimonio de muchos; y nos da las mejores esperanzas de que los trabajos por V. emprendidos con tan religioso intento, serán por extremo útiles a los estudiantes, y no dejarán de agradar a los doctos.

Deseando que así sea, Su Santidad, al dar a V. las gracias por su filial obsequio, bendice a V. cordialmente y ruega al Señor le conceda sus celestiales dones.\\
Con este mismo ánimo felicito a V. y le doy las gracias por el tomo que con tanta cortesía me ha dedicado, y me profeso de su Paternidad adictísimo:\\
R.\ CARD.\ MERRY DEL VAL
Reverendo Señor P. Ramón Ruiz Amado, S. I. en el Colegio de S.\ Ignacio.\@ -Barcelona.
\section{De los prólogos del autor}
El presente Compendio de Historia de la Iglesia sirvió durante diez años como texto para las explicaciones del Autor, impreso como manuscrito en una corta tirada, antes de darlo definitivamente a la publicidad. Solo después de este largo tiempo de prueba se ha entregado al comercio de libros, con el fin principal de poder ofrecerlo a los alumnos a un precio más reducido. La primera de las cualidades que debe tener un buen libro de texto es la claridad de estilo, y la perspicuidad en la disposición y agrupación de los hechos. Esta cualidad se ha procurado, no solo moderando el número de los párrafos, sino distinguiendo en cada uno de ellos, por la numeración y diversidad de tipos, la síntesis de toda la materia, de la más detenida explanación de ella.\\
La segunda cualidad imprescindible es el exámen crítico imparcial de los hechos, en el cual el Autor ha procurado ser escrupuloso aún en los casos en que podía resultar algo poco edificante. Así lo exige, no solo la primera ley de la Historia: la veracidad, sino aún el criterio católico; pues estamos profundamente convencidos de que la sincera exposición de los defectos reales, aún de los que se han hallado en los más elevados representantes de la Iglesia, no hace sino abrillantar más su divino esplendor y grandeza.\\
\ldots En la segunda edición se ha dado lugar a la Patrología, para satisfacer a los que no le dedican estudio especial como asignatura separada, y se ha refundido y ampliado el estudio de la Historia de las Misiones\ldots

\ldots El Autor supone que los señores profesores explanan particularmente la Historia Eclesiástica de su propio país; lo cual no es posible hacerlo en un libro de texto destinado para usarse en muy diversas regiones\ldots
En las ediciones 5.a y 6.a se han ampliado los capítulos referentes a la Historia de la Constitución de la Iglesia y de sus luchas en el terreno científico. Asimismo se ha incluído la noticia de los más recientes acaecimientos.
Solo ahora, considerando ya su obra del todo desenvuelta, ha comenzado el Autor a dar licencia para que se traduzca a lenguas extranjeras, como ya se ha hecho al italiano y se está haciendo al inglés. ¡Plega a Dios bendecir esta obra, para que produzca abundantes bienes y ofrezca un eficaz auxilio a muchos estudiantes de sagrada Teología!
Tréveris, 22 de Octubre de 1912.
\section{Abreviaturas}
\begin{itemize}
  \item AAS.\@ Acta apostolicae sedis. Commentarium officiale, Romae 1909 ss.
  \item AA.\ SS.\@ Acta sanctorum, quotquot toto orbe coluntur, ed. Bollandus et alii, Antverp. 1643 sqq., 1/?
  \item AB.\@ Analecta Bollandiana, ed, de Smedt-van Hooff-de Backer, Paris-Bruxelles 1882, 1/?
  \item ASS.\@ Acta s.\@ sedis, Romae 1881 ss. 1/40.
  \item BR.\@ Magnum Bullarium Romanum a b. Leone magno usque ad Benedictum XIII.\ ed. Cherubini, Luxenburgii 1727, 1/17 f. BRC.\ Bullarii Romani continuatio, ed. Barberi-Speccia-Secreti, Romae 1835 sqq., 1/20 f.
  \item CG.\@ Hefele-Knöpfler, Konziliengeschichte, 2. A. Freib. 1873 ff., 1/9.
  \item CL.\@ Acta et decreta s.\ conciliorum recentiorum. Collectio Lacensis, Frib. 1870/86, 1/7.
  \item CSEL.\@ Corpus scriptorum ecclesiae latinorum, ed. Vindobonae 1866 sqq. 1/?
  \item JL.\@ Regesta Pontificum Romanorum ab condita ecclesia ad a. 1198, ed. 2a curantibus Kaltenbrunner-Ewald-Löwenfeld, Lips. 1885/7, 1/2.
  \item LP.\@ Le Liber Pontificalis, Texte, introduction et commentaire par L. Duchesne Paris 1886, 1/2.
  \item MG.\@ AA.\@ Monumenta Germaniae historica, ed. Pertz-Waitz-Dümmler, Hannov.-Berol. 1826, sqq. Auctores antiquissimi.
  \item EE.\@ Epistolae.
  \item LL.\@ Leges.
  \item SS.\@ Scriptores.
  \item MGP.\@ Monumenta Germaniae paedagogica ed. Kehrbach, Berol. 1886 sqq.
  \item Mgr.\@ Monografía.

  \item PG.\@ Migne, Patrologiae cursus completus, Patrologia graeca usque ad saec, XV., Par. 1857 sqq., 1/161.

  \item PL.\@ Migne, Patrol.\ cursus completus, Patrologia latina ab aevo apostolico usque ad Innocentium III., Par. 1854 sqq. 1/221.
  \item RHE.\@ Revue d'histoire ecclésiastique, Louvain 1900 ss.
  \item RQH.\@ Revue des questions historiques, Paris 1866 ss.
\end{itemize}
\chapter{Introducción}
\subsection{Bibliografía}
\begin{itemize}
  \item C. de Smedt, Introductio generalis ad hist.\ ecclesiasticam critice tractandam, Gand. 1876. Chevalier, Répertoire des sources histor.\ du moyenâge I. Biobibliographie, 2.\ ed.\ Paris 1905; II.\ Topo-bibliographie, Montbél. 1894/1903; Potthast, Bibliotheca hist.\ medii aevi, 2.\ ed.\
  \item Berol. 1896, 1/2; Hurter, Nomenclator litterarius theologiae catholicae, 3.\ ed. Oenip. 1903 ss. 1/5; Langlois, Manuel de bibliographie hist.\ I.\ Instruments bibliograph., 2.\ éd.\ Par.\ 1901.-Dahlmann-Waitz-Steindorf, Quellenkunde d.\ deutsch.\ Gesch.\ 7.\ A.\ Lpzg.\ 1906; Ergzgsb.\ 1907.-Molinier, Les sources de l'hist.\ de France, Par. 1901/6, 1/6; Monod, Bibliographie de l'hist.\ de France\ldots jusqu'en 1789, Par. 1888.:::
  \item Gardiner-Mullinger, Introd.\ to the study of engl.\ hist., 3.\ ed. London 1894; Gross, Sources and litt.\ of engl.\ hist.\ (d. 1485), London 1900. Lozzi, Biblioteca istorica della antica e nuova Italia, Imola 1886/7. 1/2. De Smedt, Principes de la critique hist. Liège-Paris 1883; Moeller, Traité des études hist., Louvain 1889; Langlois et Seignobos, Introduction aux étud.\ hist.\ 3. éd.\ Paris 1905.
\end{itemize}
\section{Concepto de Historia Eclesiástica}
Solo las cosas que están sujetas a mudanzas tienen historia, la cual es (objetivamente) la serie de actuaciones (activas o pasivas) y de los estados consiguientes, por que una cosa ha pasado. Pero como todo lo que se actúa tiene un fin, al cual ha de tender por el desenvolvimiento de su naturaleza y actividad, este desenvolvimiento (único interesante para el espíritu humano) es el objeto propio de su historia.

El principal de los seres sujetos a semejante desenvolvimiento es el hombre, y por ende, es el más digno asunto de la historia; la cual se llama biografía cuando le considera aislado; historia particular, cuando le estudia como miembro de una sociedad limitada; e Historia universal, cuando le considera formando parte de toda la Humanidad. Mas el hombre tiene un fin doble: en cuanto se le propone un fin temporal y terreno, es objeto de la Historia profana; la cual se divide a su vez, según que estudie los varios ramos de la humana actividad, en política, jurídica, económica, literaria, etc. En cuanto el hombre tiene un fin ultraterreno (transcendental), es objeto de la Historia de la Religión, rama la más importante de toda la Historia, la cual estudia el desenvolvimiento del conocimiento y culto de la Divinidad en los diferentes pueblos. Mas, lo propio que para su fin profano, el hombre se asocia para el fin religioso, y así hallamos sociedades religiosas en todas partes donde se ha practicado la religión.

\textbf{La sociedad religiosa se confundió, en el Paganismo}, con la sociedad política, porque la vida religiosa había venido a no ser más que una parte de la vida civil. Pero el Cristianismo, destinado a ser religión universal, hubo de producir la distinción de estas dos esferas. Y así, solo en el Cristianismo hallamos una Iglesia por sí, constituída, no solo por la comunión de unas mismas creencias y prácticas cultuales, sino como verdadera sociedad religiosa, dotada de propia organización exterior. El Cristianismo, única religión verdadera y legítima, se encarna en la Iglesia fundada por Cristo; de modo que la Historia del Cristianismo se identifica substancialmente con la Historia de la Iglesia. Por esto la Historia de la Iglesia es la más noble e interesante parte de la Historia universal.\\
En la Iglesia descubrimos dos elementos: el divino, o sea, todo lo que ha recibido como dote de su divino Fundador y se le da por la asistencia del Espíritu Santo; es a saber: el tesoro de las verdades reveladas y de las gracias y medios para comunicar la gracia, los principales fundamentos de su constitución, la indefectibilidad e infalibilidad; y un elemento humano, o sea, los hombres que la constituyen y las acciones de los mismos. El elemento divino forma el alma de la Iglesia, es inmutable, y, por tanto, no tiene propiamente historia.\\
El elemento humano, por el contrario, está sujeto a mudanzas que pueden ser materia de la Historia. El elemento divino se endereza a salvar a los hombres de todos los tiempos y países; pero como ellos han de cooperar libremente a esta acción salvadora, esta cooperación, o sea, el modo como los hombres, desde la fundación de la Iglesia, se han aprovechado de sus bienes divinos, forma el propio objeto de la Historia de la Iglesia; la cual podemos definir con Möhler: «La serie de los desarrollos del principio de luz y de vida, comunicado por Cristo a la Humanidad, para volver a congraciarla con Dios y disponerla a su glorificación» 1.

La importancia de la Historia eclesiástica se colige de su mismo objeto. La Iglesia es la más grande y elevada de las obras de Dios; y, como llegamos al conocimiento de Dios por sus obras, así alcanzamos el de la Iglesia por el estudio de su Historia. Como la Historia humana interesa al hombre porque se mira en ella, así la Historia de la Iglesia interesa al cristiano, porque ve en ella el pasado de su gran familia religiosa. -Para el teólogo ofrece una viviente Apología, y en ella ve que la Iglesia actual es esencialmente la misma que al principio, a pesar de las mudanzas de los hombres, y de la extensión en el espacio y el tiempo. En ella vemos, que el florecimiento de la Iglesia depende principalmente de la piedad, sabiduría y celo de los eclesiásticos; y nos movemos a amarla, conociéndola mejor. Por el contrario, la ignorancia de la Historia eclesiástica es perniciosa para \textbf{las demás Ciencias teológicas: para la Dogmática, la Moral y Pastoral, la Exégesis, y, sobre todo, para el Derecho canónico.}
\section{División de Historia Eclesiástica}
Se divide la Historia Eclesiástica, ya objetiva, ya cronológicamente.\\
\textbf{- I}. Por la diversidad de sus funciones, se divide la Historia
de la Iglesia en externa e interna. La primera se ocupa en las relaciones de la Iglesia con las personas o sociedades que están fuera de ella; la segunda, en su actividad con los que son sus miembros.
\begin{enumerate}
  \item La Historia externa se ocupa en sus relaciones: \begin{enumerate}
          \item con las sociedades religiosas a que contradice, como única Iglesia legítima: sus relaciones con el Paganismo, Judaísmo, Islamismo (Historia de las Misiones) y con las sectas heréticas;
          \item con las sociedades políticas o Estados: cómo y hasta qué punto ha logrado el espíritu de la Iglesia penetrar en la vida política, influir en la legislación, etc.; los obstáculos que los Estados han opuesto a su acción; sus pretensiones para sojuzgar a la Iglesia y hacerla servir a sus intentos.
        \end{enumerate}
  \item La Historia interna tiene un doble asunto: \begin{enumerate}
          \item La Historia de la constitución de la Iglesia muestra de qué manera ha conservado los fundamentos puestos por Cristo (distinción entre clérigos y legos, grados jerárquicos, etc.) y los ha desenvuelto introduciendo nuevos grados en la Jerarquía, y asignando en particular a cada uno su jurisdicción.
          \item El modo cómo ha procurado conducir a cada uno a su fin último: a unos en la práctica de los consejos evangélicos (Hist.\ de la vida monástica); a todos por la administración de los Sacramentos, la concesión de indulgencias, uso de los Sacramentales y actos del culto (Hist.\ litúrgica). Asimismo, ilustrando las inteligencias con la doctrina revelada, formulando los dogmas (Hist.\ de los Dogmas), declarándolos científicamente (Hist.\ de la Ciencia eclesiástica, Patrología), predicando la doctrina (Hist.\ de la Predicación, de la Catequesis, etc.). Finalmente, legislando para determinar particularmente las acciones de los fieles (Hist.\ de la Disciplina eclesiástica y del Derecho canónico).

        \end{enumerate}
\end{enumerate}

\textbf{- II}. Cronológicamente se divide la Historia Eclesiástica en varias épocas, distintas por acaecimientos transcendentales que han cambiado las circunstancias de la existencia de la Iglesia. Los principales de esos acontecimientos son: la entrada en la Iglesia de los pueblos germánicos, y la aparición del espíritu moderno, hostil a la misma. Por ellos se divide la Historia Eclesiástica en tres grandes épocas o Edades:
\begin{enumerate}
  \item La Edad Antigua, en que la Iglesia se extiende principalmente entre los pueblos de cultura greco-romana, abarca desde su fundación hasta fines del siglo VII, y se subdivide en dos períodos, separados por la conversión de Constantino:\begin{enumerate}
          \item el primer período, desde la Fundación hasta el Edicto de Milán (313), es la época de la lucha contra el Judaísmo y el Paganismo; de las persecuciones y los apologistas.
          \item El segundo período, desde el Edicto de Milán hasta el VI Concilio universal (313 --- 680), es la época del desarrollo de la constitución y doctrina de la Iglesia: de los Concilios y los Padres: Período dogmático.
        \end{enumerate}
  \item La Edad Media es la época en que la Iglesia influye principalmente en los pueblos germánicos y eslavos, mientras el Oriente cae en el Cisma y bajo el poder del Islamismo. Las ideas cristianas penetran en toda la vida humana; la unión harmónica entre las potestades civil y eclesiástica es la más íntima. Esta época se divide en tres períodos: \begin{enumerate}
          \item Tercer período (680 --- 1073): Los pueblos germánicos entran en la Iglesia y forman sus constituciones con el magisterio de la misma: El Estado se arroga la tutela de la Iglesia.
          \item Cuarto periodo (1073 ---1307): Gregorio VII y sus sucesores ponen fin a esta tutela y elevan el Pontificado a su mayor altura.
          \item Quinto período (desde el Destierro de Aviñón, 1307, hasta el Protestantismo, 1517): El prestigio del Pontificado disminuye por efecto del Destierro y del Cisma, que fué su consecuencia: Concilios y pretensiones de reforma.
        \end{enumerate}
  \item La Edad Moderna. Una parte considerable de Europa se separa de la Iglesia y la combate; fórmanse las ideas anticatólicas y anticristianas que culminan en la negación de toda Autoridad constituída por Dios. La Revolución francesa divide esta época en dos períodos: \begin{enumerate}
          \item Sexto período, desde el Protestantismo hasta la Revolución (1517 --- 1789), tiempo de la revolución eclesiástica bajo la tiranía de los Estados.
          \item Séptimo período, desde la Revolución hasta el presente: época de las revoluciones políticas, del Estado ateo y del Laicismo. Los límites de estas épocas no siempre se fijan del mismo modo, por no haberse introducido súbitamente las mudanzas que las caracterizan. --- Algunos distinguen primero una Edad apostólica (hasta 150). La Paz de Westfalia (1648) ponen algunos como límite de un período; pero esto solo vale para Alemania. Particularmente se advierte variedad en la manera de fijar el principio y fin de la Edad Media. El primero se pone en 476, fecha de la caída del Imperio romano; pero esto solo tiene relación con el Occidente. También se hace comenzar la Edad Moderna con la aparición del Humanismo (a mediados del siglo xv) y mejor pudiera fijarse como límite el Concilio Tridentino.
        \end{enumerate}

\end{enumerate}
\section{Fuentes y ciencias auxiliares de la Historia Eclesiástica}
Llámanse fuentes de la Historia, los objetos procedentes del pasado, aptos para darnos noticia de los acaecimientos históricos; y
así, las fuentes de la Historia Eclesiástica son muy diversas y numerosas, y se dividen:
\begin{enumerate}
  \item Por su autor, en divinas (los libros del Nuevo Testamento) y humanas; y éstas a su vez, según la posición de su autor, en privadas y públicas (procedentes de personas con carácter oficial).
  \item Por su fin próximo, se dividen en narraciones y monumentos o reliquias de una acción histórica! De este número son las Actas, Decretos de Autoridades o Concilios, los documentos en sentido estricto, Concordatos, Actas de martirios, protocolos de visitas, etc., así como los objetos usuales de los antiguos tiempos.-A su vez las narraciones se subdividen en primarias, procedentes de testigos inmediatos o personas que han intervenido en las acciones y nos dan su propio juicio de ellas; y secundarias, redactadas por personas ya distantes de los sucesos, que no transmiten su propia impresión, sino la que de otros recibieron.-En razón de su importancia, ocupan generalmente el primer lugar los monumentos y las fuentes primarias.
  \item Por la forma, se dividen en escritas, monumentales (edificios, pinturas, medallas o monedas, armas, etc.) y orales (tradiciones, leyendas, mitos).
  \item Según la profesión religiosa del autor, en domésticas (de fieles) y extrañas (de enemigos o extraños).
\end{enumerate}
\subsection{Bibliografía} Para hacerlas más accesibles, las fuentes escritas se han reunido en
colecciones, de las cuales citaremos solo las más importantes.
\begin{enumerate}
  \item ESCRITORES ECLESIÁSTICOS:\@ \begin{enumerate}
          \item Migne, Patrologiae cursus completus Patrolog.\ latina ab aevo apostolico usque ad Innocentium III.\@ (1216), Paris 1854 ss.\ en Fol. 1/221;
          \item Patr.\ graeca usque ad saec.\ xv. Paris 1857 ss.\ en Fol. 1/161; Horoy, Medii aevi biblioth.\ patristica, Paris 1879 sqq.\ 1/5 (continuación de Migne, Patrol.\ lat.\@); Corpus scriptorum eccles.\@ latin.\@ Vindobonae 1866 sqq.\ 8°1/?; Die griech.\ christl. Schriftsteller d.\ ersten 3 Jhrh., Lpzg. 1897, 1/?; Hurter, Ss.\ patrum opuscula selecta ad usum praesertim studiosorum theologiae, Oeniponte 1868 sqq.\ 16° Ser.\ I.\ 1/48, Ser.\ II.\ 1/6; Rauschen, Florilegium patrist.\ Bonnae 1904 sqq.; A.\ Mai, Scriptorum veterum nova collectio, Romae 1825/38, 1/10; Idem, Spicilegium Romanum, Ibid.\ 1839/44, 1/10; Mai-Cozza, Nova patrum bibliotheca, Ibid.\ 1844 ss., 1/10; Pitra, Spicilegium Solesmense, Par.\ 1852/8, 1/4; Idem, Analecta sacra eccl.\ Par.\ 1876/83, 1/4; Idem, Anal.\ novissima, Par.\ 1885/8, 1/2; Mon.\ Germ.\ hist.\ Auctores antiquissimi, Berol.\ 1877 sqq.\ 1/?; Chabot-Guidi-Hivernat, Corpus scriptorum christ.\ orientalium, Paris 1903 ss.\ 1/?
        \end{enumerate}
  \item DECRETOS PONTIFICIOS:\@ \begin{enumerate}
          \item Textos: Constant, Pont.\ Rom.\ a s.\ Clemente I.\ usque ad Leonem Magnum Epistolae genuinae, Par.\ 1721 (Schoenemann, Götting.\ 1796); Epistolae Leonis M.\@ PL.\@ t.\ 54; A.\ Thiel, Pont.\ Rom.\ Epist.\ a s.\ Hilario ad s.\ Hormisdam, Brunsbergae 1868: Registrum Gregorii I.\ in MG.\@ EE.\@ t.\@ 1.\ ss; Rodenberg, Epist.\ Rom.\ pont.\ s.\ XIII., Berol. 1887/94, 1/3 (MG.); Los Registros de los Papas del s.\ 13.\@ y 14.\@ han sido editados desde 1883 por la École française de Rome; Loewenfeld, Ep.\ Rom.\ pont.\ ineditae, Lipsiae 1885; Pflungk-Hartung, Acta Rom.\ pont.\ inedita, Tüb.\ 1881 sqq.\ 1/3; Bullarium Romanum, ed.\ Taurini 1857 sqq.\ fol.\ 1/24; Benedicti XIV.\@ Bullarium, Romae 1754 sqq.\ fol.\ 1/4; Bullar.\ Romani continuatio (hasta Gregor. XVI incl.) ed.\ Barberi-Spezzia-Segreti, Rom.\ 1835 sqq.\ fol.\ 1/20; Extracto del Bullarium: Eisenschmidt, Acta Pii IX.\ Romae 1848/65, 1/3; Liber diurnus, ed.\ Sickel, Vindob.\ 1889; Regulae cancellariae, ed.\ Ottenthal, Lips.\ 1888.
          \item Regestas: Regesta Pontif.\ Rom.\ ab condita ecclesia ad.\ a.\ 1198, ed.\ II.\ cur. Loewenfeld-Kaltenbrunner-Ewald, Lipsiae, 1885/8 1/2 4°; Reg.\ Pont.\ Rom.\ inde ab anno 1198 ad a.\ 1304, ed.\ A.\ Potthast, Berol. 1874/5 1/2 4°; Kehr, Reg.\ pont.\ Rom.\ iubente regia soc.\ Gottingensi congessit\ldots Berol.\ 1906, 1/?— Biografías de Papas: Liber pontificalis, ed.\ Duchesne, Par.\ 1886 --- 92, 1/2.; ed.\ Mommsen (MG.) Berol.\ 1898 sqq.; Watterich.\@ Vitae pont.\ Rom.\ ab ex.\ saec.\ IX.\ usque ad finem s.\ XIII.\@ Lips.\ 1862, 1/2; Vitae pap.\ Avenionensium, ed.\ Baluzius, Par.\ 1693, 1/2; Platina, Vitae pont.\ Rom.\ Venet.\ 1479 (alcanza hasta 1471), continuado por Panvinius hasta Pío IV, y por Cicarelli hasta Clemente VIII.\@
        \end{enumerate}
  \item ACTAS DE CONCILIOS:\@  \begin{enumerate}
          \item Collectio regia, Par. 1644 sqq. 1/37 fol.; Textos: Constant, Pont.\ Rom.\ a s.\ Clemente I. usque ad Leonem Magnum Epistolae genuinae, Par.\ 1721 (Schoenemann, Götting. 1796); Epistolae Leonis M.\ PL.\ t.\ 54; A.\ Thiel, Pont.\ Rom.\ Epist.\ a s.\ Hilario ad s.\ Hormisdam, Brunsbergae 1868: Registrum Gregorii I.\ in MG.\ EE.\ t.\ 1.\ ss; Rodenberg, Epist.\ Rom.\ pont.\ s.\ XIII., Berol.\ 1887/94, 1/3 (MG.); Los Registros de los Papas del s.\ 13.\ y 14.\ han sido editados desde 1883 por la École française de Rome; Loewenfeld, Ep.\ Rom.\ pont.\ ineditae, Lipsiae 1885; Pflungk-Hartung, Acta Rom.\ pont.\ inedita, Tüb.\ 1881 sqq.\ 1/3; Bullarium Romanum, ed.\ Taurini 1857 sqq.\ fol.\ 1/24; Benedicti XIV.\ Bullarium, Romae 1754 sqq.\ fol.\ 1/4; Bullar. Romani continuatio (hasta Gregor.\@ XVI incl.) ed.\ Barberi-Spezzia-Segreti, Rom.\ 1835 sqq.\ fol.\ 1/20; Extracto del Bullarium: Eisenschmidt, Acta Pii IX.\ Romae 1848/65, 1/3; Liber diurnus, ed.\ Sickel, Vindob.\ 1889; Regulae cancellariae, ed.\ Ottenthal, Lips. 1888.
          \item Regestas: Regesta Pontif.\ Rom.\ ab condita ecclesia ad.\ a.\ 1198, ed.\ II.\ cur.\ Loewenfeld-Kaltenbrunner-Ewald, Lipsiae, 1885/8 1/2 4°; Reg.\ Pont.\ Rom.\ inde ab anno 1198 ad a.\ 1304, ed.\ A.\ Potthast, Berol.\ 1874/5 1/2 4°; Kehr, Reg.\ pont.\ Rom.\ iubente regia soc\. Gottingensi congessit\ldots Berol.\ 1906, 1/?— Biografías de Papas: Liber pontificalis, ed.\ Duchesne, Par.\ 1886 --- 92, 1/2.; ed.\ Mommsen (MG.) Berol.\ 1898 sqq.; Watterich.\ Vitae pont.\ Rom.\ ab ex.\ saec.\ IX.\ usque ad finem s.\ XIII.\ Lips.\ 1862, 1/2; Vitae pap.\ Avenionensium, ed.\ Baluzius, Par.\ 1693, 1/2; Platina, Vitae pont.\ Rom.\ Venet.\ 1479 (alcanza hasta 1471), continuado por Panvinius hasta Pío IV, y por Cicarelli hasta Clemente VIII.\
        \end{enumerate}
  \item ACTAS DE CONCILIOS:\@  \begin{enumerate}
          \item Collectio regia, Par.\ 1644 sqq.\ 1/37 fol.; Hardouin, Acta conciliorum et epist.\ decretales ac constitut.\ summor.\ pont.\ ab a.\ Chr.\ 34 usque ad a.\ 1714, Paris.\ 1714/5 1/11 fol.; I.\ D.\ Mansi, Sacror.\ concil.\ nova et amplissima collectio, Flor.\ et Venet.\ 1759/98 fol.\ 1/31 (hasta 1439), edit.\ instaur.\ Paris-Berlin 1886 ss.; Monumenta concil.\ gener.\ saec xv.\ Vindob.\ 1857/86 1/?; Collectio Lacensis, Acta et decreta s.\ concil.\ recentiorum, Friburg.\ 1870/82 1/7 4° (Desde 1682 hasta 1870); Sirmond-La-Lande, Concil.\ antiqua Galliae, Paris 1629/66 fol.\ 1/4; Tejada y Ramiro J., Colección de cánones y de todos los concilios de la Iglesia de España y de América, Madrid 1859, 1/6 fol.; Aguirre, Collect.\ max.\ CC.\ omnium Hisp.\ et Novi Orbis, 6 fl.\ 1693.\ Hadden-Stubs, Councils and eccl.\ documents to the great Britain and Irland, Oxford 1869 ss.\ 1/3; Hartzheim, Concilia Germaniae, Colon.\ 1759/90 fol.\ 1/11; Corpus iuris canonici, ed Friedberg, Lips.\ 1870/81.\ 1/2.
        \end{enumerate}
  \item LEYES CIVILES Y CONCORDATOS:\@  \begin{enumerate}
          \item Codex Theodosianus cum comm.\ Gotthofredi, ed.\ Ritter, Lipsiis 1734/75 fol.\ 1/6; Corpus iuris civilis, varias ediciones: moderna: edd.\ Mommsen, Krüger et Schöll, Berol.\ 1892/5, 1/3; Mon.\ Germ.\ hist., Leges fol.\ 1/5, 4° Sect.\ I/V;\@ E v.\ Münch, Vollständ.\ Sammlung aller ältern und neuern Konkordate, Lpz.\ 1830, 1/2; Walter, Fontes iuris eccl.\ antiqui et hodierni, Bonnae 1862; Nussi, Conventiones, Mog.\ 1870; Schneider, Fontes iuris eccl.\ novissimi, Regsb.\ 1892.
        \end{enumerate}
  \item ACTAS DE Mártires y Vidas de Santos:  \begin{enumerate}
          \item Ruinart, Acta primor.\ martyrum sincera et selecta.\ Paris.\ 1689, Ratisb.\ 1859; Le Blant, Les actes des martyrs, Supplément aux Acta sincera etc.\ Paris 1884; Assemani, Acta ss.\ martyrum orient.\ et occid.\ Romae 1748, 1/2; Hurter, Opusc.\ sel.\ t.\ 13; Acta Sanctorum ed.\ Bollandus etc.\ Antwerp.\ 1643 sqq.\ fol.\ 1/?; Analecta Bollandiana, Antwerp.\ 1882 sqq.\ 1/?; Mabillon, Acta Sanctorum ord.\ s.\ Bened.\ Paris 1668 sqq.\ fol.\ 1/9; Bedjan, Acta sanctorum et martyrum syriace, Lips.\ 1890/7, 1/7; Krusch, Passiones vitaeque sanctorum aevi Merow.\ (MG.\ SS.\ rer.\ Mer.\ t.\ 3); De Smedt et de Backer, Acta ss.\ Hiberniae, Edinb.\ 1888; Martyrologium Hieronym.\ ed.\ De Rossi et Duchesne (AA.\ SS.\ Nov.\ 2); Calendarium africanum vetus ap.\ Mabillon, Anal.\ vetera 3.\ 398.
        \end{enumerate}
  \item ESCRITOS CONFESIONALES:\@ \begin{enumerate}
          \item Denzinger, Enchiridion symbolor.\ et definit.\ quae de rebus fidei et morum a conciliis oecum.\ et summis pont.\ eman.\ 11.\ ed.\ cur.\ Tritz, Wirceb.\ 1911; Hahn, Biblioth.\ der Symbole und Glaubensregeln der alten Kirche, 3.\ A.\ Breslau 1897; Schaff, Biblioth.\ symbolica ecclesiae universalis.\ Neo-Ebor.\ 1884; Escritos confesionales de los luteranos, de J.\ A.\ Müller, Stuttgart 1848; de los reformados, C.\ W.\ Augusti, 7.\ A.\ Gütersloh 1890; de los Orientales, de C.\ J.\ Kimmel.\ Jena 1843.
        \end{enumerate}
  \item LITURGIAS Y RITUALES:\@ \begin{enumerate}
          \item J.\ A.\ Assemani, Codex liturg.\ ecclesiae universalis, Rom.\ 1748 sqq.\ 1/13; Daniel, Cod.\ lit.\ eccles.\ univ.\ Lips.\ 1847 sqq.\ 1/4; Cabrol-Leclercq, Mon.\ ecclesiae liturgica, Par.\ 1902, 1/?; Muratori, Liturgia Romana vetus, Venet.\ 1749 1/2 fol.; Mabillon, De liturg.\ gallicana.\ Paris.\ 1729; Pinius, Liturg.\ antiq.\ hispan.\ Roma 1749/50.\ 1/2; Denzinger, Ritus oriental.\ Wirceb.\ 1863/4, 1/2; Nilles, Kalendarium manuale utriusque eccl.\ orient.\ et occid.\ 2.\ ed.\ Oenip.\ 1896/7, 1/2; Chevalier, Biblioth.\ liturg.\ Par.\ 1893 ss.\ 1/4; Brightman, Liturgies Eastern and Western, Lond.\ 1896, 1/?
        \end{enumerate}
  \item REGLAS MONÁSTICAS:\@ \begin{enumerate}
          \item Codex regularum monast.\ et canon.\ ed.\ Luc.\ Holstenius, Romae 1661 fol.\ 1/4, aux.\ Brockie, Aug.\ Vind.\ 1759 fol.\ 1/6.
        \end{enumerate}
  \item INSCRIPCIONES:\@ \begin{enumerate}
          \item De Rossi, Inscriptiones christ.\ urbis Romae septimo saec.\ antiquiores, Rom.\ 1857 sqq.\ 4° 1/2; Le Blant, Inscriptions chrétiennes de la Gaule, Par.\ 1856/65 4° 1/2, 3.\ B.\ Nouveau recueil etc.\ Paris 1893; Hübner, Inscript.\ Hispaniae christ.\ Berol.\ 1871 4°. Del mismo: Inscript.\ Britaniae christ.\ Berol.\ 1876; Kraus, Die christl.\ Inschriften der Rheinlande, Freib.\ 1890 ss.\ 4° 1/2.
        \end{enumerate}
  \item Las Colecciones de fuentes para la historia de los diferentes países, suelen contener también materiales para la Historia profana:\begin{enumerate}
          \item Para Alemania: Monumenta Germaniae historica inde ab a.\ Chr.\ 500 usque ad a.\ 1500, ed.\ Pertz, Waitz et Dümmler, Han.-Berol.\ 1826 sqq.
          \item Francia: Recueil des historiens de Gaule et de France, Nouv.\ éd.\ Delisle, Par.\ 1869 ss.
          \item Italia: Muratori, Antiquit.\ italicae med.\ aevi, Med.\ 1738/42, 1/6; Idem, Rer.\ ital.\ scriptores, Med.\ 1723/51, 1/28, Nov.\ ediz.\ Città di Castello 1900 ss.; Hist.\ patriae monumenta (para el Piamonte) Aug.\ Taur.\ 1836/84, 1/27.
          \item Austria: Fontes rerum Austriac.\ Vindob.\ 1849 sqq.\ 1/?
          \item Hungría: Monum.\ Hung.\ hist.\ Pest.\ 1857 sqq.\ 1/32; Monumenta Vaticana hist.\ regni Hung.\ illustr.\ Budap.\ 1884 sqq.\ 1/?.
          \item Polonia: Script.\ rer.\ Polon.\ Cracov.\ 1873 sqq.\ 1/?; Bielowski, Mon.\ Pol.\ hist.\ Leop.\ 1864, 1/3, continuado en Mon.\ medii aevi hist.\ Cracov.\ 1874 sqq.\ 1/?; Acta hist.\ Ib.\ 1878 sqq.\ 1/?.
          \item Bélgica: Coll, des chroniques Belges, Brux.\ 1836 ss.; Coll.\ des historiens Belges.\ Brux. 1863 ss.\ 1/12; Analecta vaticano-Belgica, Romae 1906, 1/?.
          \item Inglaterra: Rer.\ Britan.\ medii aevi scriptores, Lond.\ 1858 sqq.\ 1/98; Theiner, Vet.\ mon.\ Hibernarum atque Scotorum hist.\ illustrantia, Rom.\ 1864.
          \item Oriente: Corpus script.\ historiae Byzantinae, Bonnae 1829/97, 1/50.
        \end{enumerate}
\end{enumerate}

\subsection{Ciencias auxiliares} Para España ofrece un copioso repertorio la España Sagrada, por el P. Flórez y sus continuadores los Padres M. Risco, A. Merino, J. de la Canal, agustinos; y los Sres. P. Sainz de Baranda, V. de la Fuente y C. R. Fort. Madrid, 1754 --- 1879.\ 51.\ v.\ 4.°

II.\@ Las fuentes nos ofrecen la materia de la Historia Eclesiástica; pero para elaborarla son necesarias las Ciencias auxiliares, de las cuales requiere unas el fondo y otras la forma de las fuentes.
\begin{enumerate}
  \item El fondo de las fuentes se refiere a toda la vida de la Iglesia;
        por tanto exige,\begin{enumerate}
          \item conocimiento de las Ciencias teológicas (Dogmática, Moral, Litúrgica, Derecho Canónico) que regulan la vida de la Iglesia. Es muy peligroso tratar de la Historia de la Iglesia sin conocimiento fundamental de su Dogmática.
          \item Por otra parte se necesita la Historia profana, por cuanto la vida y acción de la Iglesia está en muchas cosas condicionada por las circunstancias políticas, sociales, culturales y económicas de los pueblos. Asimismo se desenvuelve la vida de la Iglesia en el tiempo y en el espacio, por lo cual necesita auxiliarse
          \item de la Cronologia \footcite{Petavius1703DeDoctrina} de los años, meses y días, etc.,
          \item la Geografía \footcite{Chevalier1894Topo} la ilustra sobre el teatro de los sucesos, la extensión de la Iglesia y de sus partes, etc.


        \end{enumerate}
  \item La forma de las varias fuentes exige asimismo el auxilio de un gran número de ciencias. Las fuentes escritas requieren:\@ \begin{enumerate}
          \item la Filología \footcite{Stephanus1831Thesaurus}, así la clásica como la de las lenguas modernas,
          \item la Paleografía \footcite{Silvestre1839Paleographie}, que enseña los caracteres de los antiguos escritos, para determinar su época y procedencia,
          \item la Diplomática \footcite{Mabillon1681DeReDiplomatica}, que da criterio para juzgar los antiguos documentos.
          \item De las inscripciones trata la Epigrafía \footcite{Rossi1890Introductio},
          \item de las monedas la Numismática,
          \item de los sellos la Esfragística.
        \end{enumerate}
  \item Las Eras y sistemas de computar el tiempo, más importantes, son:\@ \begin{enumerate}
          \item La romana, ab Urbe condita, 753 a.\ de J-C., y según los años consulares y postconsulares;
          \item la griega según las Olimpiadas, períodos de 4 años, que comienzan desde el solsticio de 778 a 777 a.\ de J-C.;
          \item la Era hispánica usada en la Península hasta el siglo xv, comienza el 38 a.\ de J-C.;
          \item el Cyclus Indictionum (pago del censo romano), períodos de 15 años, cuyos años se numeran 1\ldots 15. La Indicción de un año de nuestra era, se halla añadiendo tres unidades y dividiendo el total por 15. El resto da la indicción del año de que se trata (0=15). Este cómputo se encuentra con particular frecuencia en los documentos de la Edad Media y desaparece desde el siglo XVI;\@
          \item el año del reinado de los emperadores, reyes y papas, éste todavía en uso;
          \item La Era del mundo, desde la Creación, según los bizantinos 5509 a.\ de J-C.; según los alejandrinos 5502, y según los judíos 3761;
          \item La Era cristiana, o del Nacimiento de Cristo, que Dionisio el Exiguo introdujo en Italia hacia 526, y más tarde se adoptó generalmente: el siglo VII en Inglaterra (llevada por los misioneros de S. Gregorio M.) y el siglo viii traída de allí a Francia y Alemania por los misioneros anglosajones (Wilibrordo, Bonifacio). El principio del año se ha tomado de diversas maneras: \begin{enumerate}
                  \item en 1 de Enero,
                  \item en 1 de Marzo (Rusia),
                  \item en 1 de Septiembre (Constantinopla),
                  \item en Navidad,
                  \item en 25 de Marzo (Encarnación), y esto de dos maneras: ya comenzando el 25 de Marzo anterior al Nacimiento del Señor (Calculus Pisanus), ya el 25 de Marzo que siguió al Nacimiento (Calculus Florentinus);
                  \item en Pascua. -Hasta el siglo XVI no se hizo general comenzar el 1 de Enero. Desde la reforma del Calendario hecha por Julio César (46 a. J-C.) se contó el año en Roma de 365 / días, por lo cual, después de cada tres años, se añadía uno de 366 o bisiesto. Así duró hasta Gregorio XIII.\@ Mas como esa cuenta suponía el año `11' y `12' más largo de lo que es, el siglo XVI el equinoccio de primavera que, según el Calendario Juliano, ha de ser a 21 Marzo, caía 10 días después. Por eso se introdujo la Corrección gregoriana, haciendo que al 4 de Octubre de 1582 siguiera inmediatamente el 15 de Octubre, y que en adelante sólo fueran bisiestos los años expresados por centenas cuando éstas eran divisibles por 4; de suerte que cada 400 años se suprimen tres bisiestos del Calendario Juliano. Esta corrección no fué adoptada por los protestantes hasta el siglo XVIII, y los cismáticos griegos y rusos no la han admitido todavía.

                        \textit{Nota de la reedición (Trejo, Walbert 2025): Todos los paises aceptan el calendario Gregoriano para usos civiles desde 1923, siendo Grecia el último en aplicarlo. Mas para los fines litúrgicos, algunas denominaciones cismáticas orientales utilizan el calendario Gregoriano y otras mantienen el Juliano.}
                \end{enumerate}
        \end{enumerate}
\end{enumerate}
\section{Método de la Historia Eclesiástica}
«Illud imprimis scribentium obversetur animo, primam esse historiae legem, ne quid falsi dicere audeat, deinde, ne quid veri non audeat; ne qua suspicio gratiae sit in scribendo, ne qua simultatis.» En estas palabras de su célebre Carta sobre la manera de escribir la Historia, el Papa León XIII estableció, tomándola de Cicerón (De orat. II, 15), la ley suprema a que debe obedecer la Historia, cuyo deber es decir la verdad, sólo la verdad y toda la verdad; y esto incumbe más particularmente a la Historia Eclesiástica, la cual ha de servir al Reino de la verdad, que es la Iglesia de Cristo. Así que, ni sus inclinaciones, ni ajenos deseos, le han de separar de la más estricta imparcialidad. Por lo cual: \begin{enumerate}
  \item Debe ser crítico, para separar lo falso, que puede hallarse en las fuentes legítimas, y excluir las fuentes que no lo sean.\begin{enumerate}
          \item Ha de subir hasta las fuentes y examinar su legitimidad, y la competencia y veracidad de los testimonios (Crítica de las fuentes);
          \item ha de examinar la posibilidad y verosimilitud de los hechos, rechazando los que son imposibles, por contradecir a verdades ciertas, u oponerse a las circunstancias de los tiempos, lugares y personas. Para esto necesita disponer de todo el aparato de la Critica histórica.
        \end{enumerate}
  \item La Historia ha de ser pragmática, pues los hechos no existen sino con dependencia de sus causas, las cuales han de ponerse de manifiesto, descubriendo las ideas, fines y móviles de los agentes (Pragmatismo filosófico); y sobre todo, la finalidad que proviene de la dirección de Dios en la Historia, particularmente de su Iglesia (Pragmatismo teológico). De este modo la Historia pone ante los ojos la acción de Dios y de los hombres.
\end{enumerate}

La objetividad, en que tanto se insiste ahora, no es sino la veracidad. Objetiva es la narración que refleja fielmente los hechos; subjetiva, por el contrario, la que los desfigura o colorea con los prejuicios del escritor. -Los incrédulos suelen exigirnos que escribamos sin presupuestos. Con este nombre se designan, en una ciencia, las verdades que la misma ciencia no demuestra; y en este sentido, ninguna ciencia hay que no los tenga, pues ninguna demuestra los principios evidentes, ni los que toma de otra ciencia superior. Lo único que se puede exigir razonablemente es, que el historiador no tergiverse ni disimule los hechos que parecen ofender sus ideas o creencias religiosas; como sería negarse a confesar las debilidades o daños que han oscurecido en determinadas épocas el elemento humano de la Iglesia. Cuanto el historiador católico está más seguro de la divinidad de la Iglesia, tanto estará más libre de la tentación de disimular las culpas de los hombres que en ella han intervenido; pues sabe que, sobre esas miserias humanas, ha de resplandecer con brillo mayor el elemento divino\footnote{Incorrupta rerum gestarum monumenta\ldots per se ipso Ecclesiam et Pontificatum sponte magnificeque defendunt. Leo XIII.\ 1.\ c.}.
\section{Desenvolvimiento de la Historia Eclesiástica}
\begin{enumerate}
  \item En la Antigüedad \footcite{Bardenhewer1901Patrologia}. La época de las persecuciones no era, naturalmente, a propósito para escribir historia, y los únicos trabajos que de entonces datan son las historias apócrifas de los Apóstoles, las Actas de martirios y las Crónicas de Julio Africano e Hipólito, de las que no se conservan sino fragmentos\footnote{Acerca la obra de Hegesippo v.\ § 22, 2 b.}. -El Padre de la Historia Eclesiástica es Eusebio de Cesarea de Palestina (m. 340), el cual, sobre el fundamento de la Cronografía del «Africano», escribió sus libros de Historia universal, especie de Crónica que se propone concordar la Historia universal con la bíblica y eclesiástica. Partiendo de aquí, compuso luego los diez libros de su Historia Eclesiástica, que llega hasta el a.\ 324 y es de inestimable valor por contener literal, aunque fragmentariamente, numerosas fuentes ahora perdidas. Cien años después se escribieron casi al mismo tiempo tres diversas continuaciones de ella, con el mismo título de Historia Eclesiástica, a saber: la de Sócrates, que comprende desde 305 a-439; la de Sozomeno, desde 324 a 425 (ambos fueron abogados de Constantinopla)\footnote{Ambas en PG.\ v.\ 67.} y de Teodoreto de Cyro, en Siria, desde 320 hasta 428\footnote{PG.\ 82.881/1280.}. Otro Teodoreto, lector de Constantinopla, hizo un extracto de los tres mencionados y lo continuó hasta Justino I (518 --- 527), pero de la última parte de su trabajo no conservamos más que fragmentos. Otra continuación escribió el abogado de Antioquía Evagrio desde 431 hasta 594, al cual debemos las mejores noticias sobre las controversias nestoriana y monofisita\ \footnote{PG.\ 86, 2, 2415/2886.} De menos importancia es la obra del retórico Zacarias, conservada sólo en sirio\footnote{Land, Anécdota Syriaca (Lugd.\ Bat.\ 1870), v.\ 3.\ Se han perdido la Historia cristiana del diácono Filipo de Side (Pamfilia), la Hist.\ ecles.\ del bresbítero Hesyquio de Jerusalén y las obras heréticas del ob.\ Timoteo de Berilo, del ob.\ Sabino de Heraclea (Hist.\ de los Concilios) y de Filostorgio. Sobre los trabajos históricos de los Armenios y Syrios, v.\ § 42, 6 y 7.}. Desde el año 500 hallamos en Constantinopla una serie de historiadores llamados Bizantinos \footcite{Niebuhr1829Corpus}, de los que es el más notable Nicéforo Calisto (m. 1341). De la primera mitad del siglo vii procede el Chronikon paschale s. Alexandrinum\footnote{Editado por L.\@ Dindorf, Bonn 1832, 1/2 y in PG.\ v.\ 92.}, el cual sólo tiene grande importancia para el siglo vii. La Historia de las herejías la escribieron Epifanio de Salamina y Teodoreto de Cyro en su Compendio de las fábulas heréticas.\\ En Occidente, Rufino refundió en latín la Historia Eclesiástica de Eusebio y la continuó hasta 395; pero es en muchas cosas inexacto. San Jerónimo radujo la Crónica de Eusebio y continuóla hasta 378. Dieron continuaciones de esta obra San Próspero de Aquitania, Victor de Tunun, San Isidoro de Sevilla y San Beda. Con tendencia apologética escribieron crónicas universales Sulpicio Severo (hasta 400) y Paulo Orosio, estimulado por San Agustín (§ 42, n. 9). Cassiodoro hizo en su «Historia Tripartita» un extracto de Sócrates, Sozomeno y Teodoreto, y lo continuó hasta 518. La Edad Media sacó generalmente de Rufino y Cassiodoro, sus conocimientos sobre la Antigüedad eclesiástica. La primera Historia literaria fue el «Lib.\ de viris illustribus», compuesto por San Jerónimo y continuado por Gennadio, San Isidoro de Sevilla y San Ildefonso de Toledo.
  \item En la Edad Media la Historia Eclesiástica se mezcla con la profana, y hasta el siglo XIII hallamos un número regular de Crónicas universales, divididas generalmente en las seis edades del mundo, a ejemplo de Beda el Venerable. Comúnmente ponen al principio la Crónica de Eusebio continuada por San Jerónimo hasta 378, o un extracto de la misma, y ofrecen por lo general áridos extractos de los documentos históricos hasta el tiempo del autor. Estas continuaciones y los demás escritos históricos, se limitan sustancialmente a la historia particular de un pueblo o región (obispado, ciudad, monasterio), y se escribieron en general en forma de anales. Más adelante (desde el siglo XIII) el impulso dado a las Ciencias teológicas, la difusión de los conocimientos científicos y la necesidad de instrucción enciclopédica, sentida por las Órdenes mendicantes, condujeron a la recopilación de los materiales históricos y formación de obras generales. Pero se admitieron en ellas muchas leyendas y falsedades, por la dificultad del comercio literario y otros motivos\footnote{PL.\ 23.\ 601/720.}.

        Crónicas nos dejaron San Isidoro de Sevilla (PL.\@ 83), Beda el Venerable (PL.\@ 95), Regino de Prüm, m. 915 (PL.\@ 132, MG.\@ SS.\@ 1.\@ 537, ss.), Herman Contracto, m. 1054 (PL.\@ 143, MG.\@ SS.\@ 5.\@ 67 ss.), Lamberto de Hersfeld, m.\@ 1080 (PL.\@ 146, MG.\@ SS.\@ 1, 3 y 5), Sigeberto de Gemblours, m.\@ 1112 (PL.\@ 160, MG.\@ SS.\@ 6.\@ 260), Otón de Frisinga, m.\@ 1158 (PL.\@ 189, MG.\@ SS.\@ 20.\@ 83 ss.) y otros. Gregorio de Tours (m.\@ 594) escribió la Historia de los Francos (PL.\@ 71, MG.\@ SS.\@ rer.\@ Mer.\@ v.\@ 1.), continuada por Fredegario; Beda la Histor.\@ gentis Anglorum (PL.\@ 95); Isidoro de Sevilla la Histor.\@ gentis Visigothorum (PL.\@ 83), Pablo Warnefrido (m.\@ 799) la Histor.\@ gentis Langobardorum (PL.\@ 95), Flodoardo (m.\@ 966) la Histor.\@ Rhemensis (PL.\@ 135, MG.\@ SS.\@ 13), Adam de Brema la Gesta pontif.\@ Hammaburgens.\@ (PL.\@ 146). Para la Historia de los Papas es de grande importancia el Liber Pontificalis \footcite{Duchesne1884Liber}, colección de biografías de los Papas, cuya primera redacción se coloca a principios del siglo vi, la segunda termina el 687, la tercera el 715; todo él fué refundido y continuado hasta 882. Son generales la Historia Eclesiástica de Haimón de Halberstad (PL.\@ 116), la Historia tripartita de Anastasio bibliotecario (PL.\@ 108), la Historia Eclesiástica de Orderico Vital (m.\@ c.\@ 1142.\@ PL.\@ 188), la de Ptolomeo de Lucca\footnote{Bei Muratori, Rer.\ ital.\ script.\ 11.\ 751 ss.}, discípulo de Santo Tomás de Aquino, y la Summa historialis de San Antonino de Florencia (m. 1459)\footnote{Ed.\ Norimb.\ 1484, Lugd.\ 1512.},
  \item Edad Moderna. A fines de la Edad Media tomó nuevo impulso la Historia, facilitada por la imprenta, estimulada por el Humanismo, que cultivó la Filología griega, latina y hebraica, y acendró el gusto; de suerte que, desde 1450 se desplegó una nueva actividad, publicándose numerosas fuentes y trabajos históricos, vgr., los de Hartmann Schedel, Juan Tritemius, Alb. Krantz, Beatus Rhenanus, Jacobo Wimpfeling, etc. La revuelta protestante interrumpió este progreso, atrayendo todas las fuerzas a la controversia teológica. Mas como el Protestantismo pretendía que la Iglesia había degenerado de su primitiva pureza, y había sido desnaturalizada por los Papas, esto reavivó los estudios históricos. Las Centurias de Magdeburgo \footcite{Magdeb1559Eccles}, obra de un grupo de teólogos protestantes presididos por Flaccio Ilírico, cuya finalidad era descubrir los comienzos, progresos y planes del Antecristo, esto es, del Papado, trazó una caricatura de la Historia Eclesiástica, contra la cual se dirigieron principalmente los Annales Ecclesiastici del Oratoriano Cardenal César Baronio, fruto de 30 años de trabajo.\\
        Por mucho tiempo estas dos obras fueron el arsenal de uno y otro partido. En el siglo XVII se hicieron en Francia notables trabajos. Los Maurinos, Oratorianos y Jesuitas sacaron a luz abundantes materiales históricos, trataron de un modo perfecto algunas partes de la Historia Eclesiástica, y establecieron (aunque imperfectamente) las leyes del método histórico. Natal Alexandre, Tillemont y Claudio Fleury expusieron toda la Historia de la Iglesia, y el genial Bossuet desenvolvió por manera brillante el concepto cristiano de la Historia. Muy inferiores a éstas son las obras de vulgarización de los historiadores franceses, Berault-Bercastel\footnote{Hist.\ de l'égl., Paris 1778/91, 12° 1/24, Toulouse, 1809, 1/12, Par.\ 1830, 1/12.}, Rohrbacher\footnote{Hist.\ universelle de l'église cath.\ Paris 1842, 8º 1/29.\ ed.\ 6.e 1870, 4º 1/29.} y Darras\footnote{Paris 1862/88, 1/44.}. En Italia, el siglo XVIII produjo la notable obra del dominico Orsi\footcite{OrsiBecchetti1752Storia} (m. 1761) que abrazaba los seis primeros siglos y fué continuada por Becchetti.

        Alemania, desolada por las guerras religiosas, vivió, el siglo XVII, de traducciones de obras francesas, y aun en el siglo XVIII sólo en la historia particular hizo trabajos estimables. El espíritu josefino-galicano hacía imposible a los católicos tratar debidamente la Historia Eclesiástica, mientras los protestantes estaban aún obsesionados por el espectro de los Centuriadores, y el Racionalismo se manifestaba ya en este terreno. Solo en el siglo XIX, los románticos promovieron el renacimiento de la Historia, favorecido por la edición de numerosas fuentes y la clara noción del verdadero método. Comenzó el Conde Fr.\ v.\ Stolberg (m. 1819) con su Historia de la Religión de Jesucristo; J.\ Ad.\ Möhler procuró juntar el criterio católico con el estudio de las fuentes, y tuvo por sucesores a Döllinger, Hefele, Alzog, De Rossi y otros. Las obras más importantes de nuestra época son la Historia de los Concilios de Hefele, y la de los Papas de Grisar y Pastor. Aun los protestantes se han esforzado, hasta cierto punto, por tratar con justicia a la Iglesia católica y su Historia.
\end{enumerate}
\begin{enumerate}
  \item Católicos. La obra de Baronio \footcite{Baronio1588Annales} comprende hasta 1198, 100 años en cada tomo, y ofrece una enorme cantidad de documentos. Los franciscanos franceses Ant.\ y Fr.\ Pagi (Antv.\ 1705, 4 tom.\ fol.) escribieron una crítica y complemento de ella. Continuaron a Baronio el ob.\ Spondanus de Pamiers y el dominio Abrah.\ Bzovius, y mejor Orderico Raynaldo, desde 1198 hasta 1565. Mansi (Luccae 1738/59 fol.\ 1/38) dió la mejor edición de Baronio y Raynaldo, valiéndose también de Pagi. Los oratorianos Laderchi y Agust.\ Theiner, han continuado la obra hasta 1572 y hasta 1585. El libro de Natal Alexandre \footcite{Natal1676Selecta} fue puesto en el Indice por su espíritu galicano, del cual le procuró librar Roncaglia (1734) con anotaciones y disertaciones. El de Fleury (m.\ 1723) llega hasta 1414 \footcite{FleuryFabre1691Histoire} y fué continuado por Cl.\ Fabre. -Sebastián Le Nain de Tillemont \footcite{Tillemont1693Mem} formó un artificioso mosaico de fuentes, y sólo llega hasta 513. Obras particulares de Historia Eclesiástica son:\begin{enumerate}
          \item en Italia, Ughelli; Italia sacra, Roma 1644.\ f.\ 1/9, aumentada por Coleti (Venet.\ 1707/25, 1/10);
          \item en España, Florez, España sagrada, Madrid 1747 (con la continuación 46); Menéndez y Pelayo, Hist.\ de los Heterodoxos españoles, 2.ª\ ed.\ 1911, t.\ 1.º\
          \item en Inglaterra, J.\ Lingard, The Antiquites of the Anglo-Saxon Church, 1831, 1/2; Hist.\ of England;
          \item en Francia: Gallia christiana in provincias ecclesiasticas distributa, Paris 1715 --- 1865 fol.\ 1/16.\ nov.\ edit.\ Parisiis 1875 sqq.;
          \item en Alemania, Marc.\ Hansiz, Germania sacra, Aug.\ Vindel.\ 1737/54 fol.\ 1/3; Ussermann, Episcopatus Wirceburgensis, S.\ Blas.\ 1794, Episc.\ Bamberg.\ 1802; Neugart, Episcop.\ Constant.\ Freib.\ 1803/62, 1/2; Brower-Masen, Annales Trevirenses, Leod.\ 1620 fol.\ 1/2; Hontheim, Histor.\ diplom.\ Trever.\ Aug.\ Vind.\ 1750 ss.\ fol.\ 1/3, Prodromus hist.\ Trev.\ Ib.\ 1757 fol.\ 1/2.
        \end{enumerate}
        Entre los Compendios merecen mencionarse los de Th. Katerkamp, Döllinger, Alzog, Hergenrother, Kraus, Brück, Knöpfler, etc. También el Diccionario eclesiástico de Wetzer-Welte se ha de mencionar por sus artículos de Historia Eclesiástica. En Francia ha publicado Battifol la notable colección de monografías Bibliothèque de l'enseignement d'hist.\ ecclés.\ Paris 1898 ss.
  \item Entre los protestantes han hecho trabajos de consideración, Hottinger, J.\ Basnage, contra Bossuet, y su hermano Samuel, contra Baronio; el canciller de Gottinga, Mosheim, y su discípulo M. Schröckh, profesor de Wittenberg (45 tomos). Walch escribió una historia de las herejías en once tomos. Más objetivos que éstos son Planck (m.\ 1832), Marheinecke y Aug. Neander (m.\ 1850). Desde J.\ Salomón Semler (m. 1791) el Racionalismo se introdujo en la Historia y domina hasta el presente, esforzándose por explicar la vida de la Iglesia por causas naturales, particularmente en la Historia de los Dogmas y la antigua Historia literaria. Entre sus representantes descuellan Fr.\ Chr.\ Baur (m.\ 1860), Hilgenfeld y Adolfo Harnack, el cual ha alcanzado prestigio, sobre todo en la Historia de la Antigua Literatura cristiana \footcite{HilgenfeldHarnack1884Dogmengeschichte}. Con más positiva tendencia le ha seguido en este terreno Th.\ Zahn\footnote{Forsch.\ z.\ Gesch.\ des neutestl.\ Kanons, Erlangen, Lpzg.\ 1881 ff.\ 1/7.}. Merece mencionarse en este lugar, la Real-Enciclopedia de la Teología e Iglesia protestante\ \footnote{Alb.\ Hauck, 3.\ A.\ Lpzg.\ 1896 ff.\ 1/22.}.
\end{enumerate}

Si nos preguntamos por el estado actual de los estudios de Historia de la Iglesia, hemos de contestar: que la Historia de las herejías, de las relaciones de la Iglesia y del Estado, la de las grandes acciones de la Iglesia, y la de los Papas (cuando se termine la obra de Grisar y Pastor), se han estudiado satisfactoriamente. Por el contrario, se conoce todavía relativamente poco la Historia de la Iglesia en la Cura de almas, en la enseñanza (si no es de las Universidades), en las obras sociales y populares, y la Historia de las ciencias eclesiásticas, excepto la de los Dogmas y de la Predicación y Catequesis. Esta ignorancia es especial respecto de los siglos X, XI, XIV y xv, y el período que va desde la Paz de Westfalia hasta la Revolución francesa. Por falta de suficiente estudio de las circunstancias religiosas, morales y eclesiásticas del fin de la Edad Media, no se ha dado hasta ahora una explicación satisfactoria de la rápida propagación del Protestantismo. También reclama atención la Historia de las Misiones en la época moderna. Ahora despierta grande interés la Historia Universal de las Religiones, al cual contribuyen, por una parte, los nuevos conocimientos adquiridos por los misioneros sobre las supersticiones de los salvajes y los descubrimientos hechos en Oriente. Pero, por otra parte, no hay que perder de vista que muchos fomentan ese estudio con la esperanza de involucrar la Historia de la Religión cristiana en el común desenvolvimiento de las ideas religiosas en el mundo. Hasta ahora son más los trabajos particulares que las exposiciones de conjunto \footcite{Saussaye1898Religionsgeschichte}. Pero en todo caso, ningún historiador sensato puede esperar que la Historia de la Iglesia llegue a reducirse a un capítulo de la Historia de las Religiones \footcite{Shrors1905Nicht}.
\part{Época Pimera. Primer Período: de las persecuciones del Cristianismo hata el edicto de Constantino}
La primera Comunidad cristiana nació el día de Pentecostés, en Jerusalén, y se aumentó rápidamente bajo la dirección de los Apóstoles. La persecución que le movió la Autoridad israelita, obligó a algunos fieles a esparcirse por Samaría y Galilea, y ellos y los Apóstoles y diáconos convirtieron una parte de Palestina. A poco fué admitido en la Iglesia el primer gentil, se fundó en Antioquía la primera comunidad de conversos del Paganismo, y poco a poco se fué divorciando la Iglesia de la Sinagoga. Los Apóstoles se dirigieron a todas las regiones y predicaron el Evangelio en las ciudades, desde donde se extendió a las aldeas, propagándose, hasta la muerte del postrer Apóstol (c. 100), no sólo por Palestina, Siria y el Asia Menor, sino en la Mesopotamia, Armenia, Persia, Arabia, y acaso también en la India; por la Península Balkánica, Italia y España. Durante lo más recio de las persecuciones, por su sola fuerza interna, penetró en todo el Imperio Romano, el SO.\@ de Asia, el Norte de África, el Sud de Europa hasta el Danubio, y el Este hasta el Rhin y el Mar del Norte. Tolerado al principio como «secta judaica», fué luego perseguido por el Estado, hasta que triunfó del Poder civil en una lucha de tres siglos. Al principio se predicó la doctrina cristiana, en forma de narración de la Vida de Cristo, para excitar la fe. Pero luego que los fieles procuraron darse razón del contenido de ésta, hubieron de distinguir sus creencias del Judaísmo y del Paganismo (en lo cual erraron algunos: los Gnósticos), y desentrañar sus propios dogmas; donde erraron otros, como los Antitrinitarios, Montanistas, etc. La Ciencia eclesiástica se opuso a los perseguidores y a los herejes, por lo cual revistió forma apologética o polémica. El gobierno de la Iglesia estuvo al principio en manos de los Apóstoles, los cuales confiaron luego a los diáconos la administración de las cosas temporales; pero la fundación de nuevas comunidades exigió el nombramiento de Obispos, y cuando las iglesias crecieron, fueron necesarios los presbíteros para auxiliar al Obispo. En este período aparecen asimismo los grados inferiores del Clero, y la subordinación de los Obispos bajo los Arzobispos, Patriarcas, etc.; de suerte que, hacia el fin de él, hallamos ya completa la Jerarquía eclesiástica, por más que no estuvieran todavía fijamente establecidas las facultades de algunos de sus grados. El desenvolvimiento del Derecho eclesiástico escrito se limitó sustancialmente a la Disciplina penitencial (que adquiere forma fija en el siglo III) y las irregularidades.
\chapter{Capítulo Primero: Estado religioso y social del mundo, a la aparición del Cristianismo}
El Cristianismo no es fruto del mero desenvolvimiento humano e histórico; sino tiene origen y virtud sobrenaturales. Mas aun cuando las circunstancias naturales que acompañaron su aparición, no basten para explicarla, es con todo importante conocerlas, por el influjo que en su propagación ejercieron. Desde el punto de vista religioso, el mundo se dividía entonces entre el Judaísmo y el Paganismo.
\section{Estado del Judaísmo}
\section{Estado del paganismo en el Imperio Romano}
\subsection{Preparación de los paganos para el Cristianismo}
\subsection{Obstáculos para la propagación del Cristianismo}
\section{JesuCristo, Fundador de la Iglesia}
\section{La primitiva comunidad de Jerusalem}
\section{La Iglesia se separa de la Sinagoga. Admisión de los Gentiles}
\section{El apostol de las gentes: San Pablo}
\section{San Pedro. Fundación de la Iglesia Romana}
\section{Los demás apóstoles: Discípulos de los apostoles}
\section{Causas y carácter de las persecuciones}
\section{Las persecuciones}
\section{Significado del Martirio para la Iglesia}
\section{El Cristianismo atacado con armas intelectuales}
\section{Extensión del Cristianismo al fin de las persecuciones}
\chapter{Capítulo Segundo: Desarrollo de la Doctrina Eclesiástica. Herejías}
\section{Herejes judaizantes}
\section{Gnosticismo}
\section{Los monarquianos}
\section{Ciencia y literatura cristiana}
\section{Desarrollo de la Doctrina Eclesiástica}
\chapter{Capítulo Tercero: Constitución de la Iglesia. Culto y Disciplina: rasgos fundamentales de la constitución eclesiástica}
\section{Desarrollo de la constitución}
\section{Vida del clero}
\section{Primado del Obispo de Roma}
\section{Bautismo y Confirmación. Controversia sobre el bautismo de los herejes}
\section{La disciplina penitencial. Cismas}
\section{El culto. Las fiestas. Controversia sobre la Pascua}
\section{La vida cristiana}
\part{Época Primera. Segundo Período: Época de luchas Dogmáticas}
\chapter{Capítulo Primero: historia externa de la Iglesia}
\section{Cristianismo fuera del Imperio Romano}
\section{Ruina del paganismo en el Imperio Romano}
\section{La Iglesia y el Imperio Romano}
\chapter{Capítulo Segundo: Desenvolvimiento científico. Herejías y concilios}
\section{Los donatistas}
\section{El arrianismo. Primer Consilio Universal}
\section{Derivaciones del Arrianismo y otras herejías simultáneas}
\section{El priscilianismo. Sectas menores}
\section{El pelagianismo}
\section{El nestorianismo}
\section{El monofisitismo}
\section{El monotelismo}
\section{Ciencia y literatura Eclesiástica}
\chapter{Capítulo Tercero: La constitución Eclesiástica}
\section{Perfeccionamiento y multiplicación de los oficios eclesiásticos}
\section{Primado del Obispo de Roma}
\section{Los Sínodos}
\section{El Clero}
\section{Vida monástica}
\chapter{Capítulo Cuarto: Culto, disciplina y vida cristiana}
\section{Sacramentos: La Santa Misa}
\section{Derecho penal. Penitencia pública}
\section{Templos, ayunos, festividades religiosas}
\section{Culto de los Santos y sus reliquias. Romerías}
\section{La vida cristiana}
\part{Época Segunda. Tercer Período: Desde la irrupción de los bárbaros hasta el pontificado de San Gregorio VII}
\chapter{Capítulo Primero: Extensión y limitación de la Iglesia}
\section{Las incursiones de los bárbaros y sus próximos efectos}
\section{Conversión de los Francos}
\section{El Cristianismo en las islas Británicas}
\section{Conversión de Alemania}
\section{Conversión de los pueblos Escandinavos}
\section{Conversión de los Eslavos y los Húngaros}
\section{El Islam como enemigo de la Cristiandad}
\chapter{Capítulo Segundo: El Pontificado y el Imperio. Estado y la Iglesia}
\section{Formación del Estado de la Iglesia}
\section{Restablecimiento del Imperio de Occidente}
\section{El pontificado y el Imperio desde San CarloMagno hasta San Gregorio VII}
\section{El Estado y la Iglesia en los reinos germánicos}
\chapter{Capítulo Tercero: Desarrollo de la Doctrina. Herejía y Cismas}
\section{Los iconoclastas y el VII Consilio Universal}
\section{El Cisma Griego. El VIII Consilio Universal}
\section{Controversias dogmáticas en Occidente}
\section{La Ciencia cristiana}
\chapter{Capítulo Cuarto: Constitución, disciplina, culto, vida cristiana}
\section{La Jerarquía. Erección de diócesis}
\section{Colecciones de Cánones. El Pseudo-Isidoro}
\section{El Clero. La vida monástica}
\section{El culto. Veneración de los Santos}
\section{Disciplina y vida cristiana}
\part{Período Cuarto: Florecimiento de la Iglesia en la Edad Media (1073 --- 1307)}
\chapter{Capítulo Primero: El Pontificado y el Imperio. El Estado y la Iglesia}
\section{Las ideas gregorianas}
\section{Conatos de reforma anteriores a San Gregorio VII}
\section{La contienda de las investiduras. San Gregorio VII}
\section{Consecuencias de la contiendad de las investiduras. X Consilio Universal}
\section{Lucha de los Emperadores de la Casa de Suabia con los papas. Inocencio III}
\section{Contiendas eclesiásticas en Inglaterra}
\section{El Pontificado cae bajo la influencia francesa}
\chapter{Capítulo Segundo: Historia externa de  la Iglesia}
\section{La lucha contra el Islamismo en Europa}
\section{Las Cruzadas}
\section{Lucha contra el Paganismo. Las Misiones}
\section{Los judíos en la edad media}
\chapter{Capítulo Tercero: Desarrollo de la vida monástica}
\section{Nuevas Órdenes con reglas antiguas}
\section{Las dos grandes Órdenes Mendicantes}
\section{Las Órdenes Militares}
\chapter{Capítulo Cuarto: La Doctrina de la Iglesia y sus adversarios}
\section{Impugnaciones de la Iglesia. Las sectas}
\section{La Inquisición}
\section{La Ciencia eclesiástica: Escolástica y Mística}
\chapter{Capítulo Quinto: Constitución, culto, vida cristiana}
\section{Desarrollo de la Constitución Eclesiástica}
\section{El culto. Los Sacramentos}
\section{El Arte cristiano}
\section{Vida y disciplina}
\part{Periodo Quinto: Obscurecimiento de la Autoridad Papal. Decadencia del Imperio (1307 --- 1517)}
\chapter{Capítulo Primero: Historia del Pontificado}
\section{El destierro de Aviñón}
\section{Lucha del Papa con Luis de Baviera}
\section{Consecuencias del Cisma. Consilio de Basiela}
\section{Cisma de Occidente. Consilio de Constanza}
\section{Los papas de la época del renacimiento}
\chapter{Capítulo Segundo. Desenvolvimiento de la Doctrina. Herejías. Ciencia Eclesiástica}
\section{La Ciencia eclesiástica}
\section{El Renacimiento literario y la Iglesia}
\section{Juan Wiclef y Juan Hus. Precursores de la Reforma}
\chapter{Capítulo Tercero: Constitución, culto y vida cristiana}
\section{La Sede Apostólica}
\section{La vida monástica. Conatos de Reforma}
\section{El clero secular}
\section{El arte cristiano}
\section{Culto, disciplina y vida cristiana}
\part{Época Tercera: Edad moderna. Periodo Sexto. De la revolución religiosa y del absolutismo del Estado (1517 --- 1588)}
\chapter{Capítulo Primero}
\section{El Protestantismo en Alemania y Suiza}
\section{Causas de la gran difusión del Protestantismo}
\section{Principios de la escición religiosa}
\section{Excomunión de Lutero}
\section{Lutero y la Revolución. Sus partidarios y adversarios}
\section{El Imperio y los novadores. División política de Alemania}
\section{Organización de las iglesias territoriales. Los príncipes}
\section{La reforma en la Suiza alemana, Zuinglio}
\section{La Dieta de Augsburgo de 1530. Tentativas de concordia}
\section{Propagación del Protestantismo en los Estados de la Liga de Schmalkalda}
\section{Carácter y últimos días de Lutero}
\section{La paz religiosa de Augsburgo}
\section{El Protestantismo en la Suiza francesa}
\section{Calvino}
\section{Guerras religiosas hasta la paz de Westfalia}
\chapter{Capítulo Segundo}
\section{La pseudo-reforma en los demás países}
\section{El Protestantismo en Francia}
\section{La reforma en los Países Bajos}
\section{La reforma en las Islas Británicas}
\section{Apostasía de los reinos del Norte}
\section{La reforma en los países orientales de Europa}
\section{Lucha contra el Protestantismo en Europa Meridional}
\section{Galileo}
\chapter{Capítulo Tercero: Desenvolvimiento interno del Protestantismo}
\section{Controversias entre las sectas}
\section{Constitución, culto y vida protestante}
\chapter{Capítulo Cuarto: La Iglesia Católica y la verdadera Reforma}
\section{Acción de los papas}
\section{El Concilio de Trento}
\section{La Compañía de Jesús}
\section{Las demás Órdenes y Congregaciones}
\section{Restauración del Catolisismo en Alemania}
\section{Renovación de las Ciencias Eclesiásticas}
\section{Controversias doctrinales. El Jansenismo}
\section{Las misiones extranjeras}
\section{Arte cristiano}
\section{Culto, disciplina, vida cristiana}
\chapter{Capítulo Quinto: Preparación de la época revolucionaria}
\section{Conatos revolucionarios en la Iglesia}
\section{Los Papas de la época del absolutismo}
\section{Incredulidad y Filo-sofismo}
\part{Época de las revoluciones: Desde las revolución Francesa hasta nuestros días}
\chapter{Capítulo Primero: La Revolución Francesa. Sus consecuencias próximas}
\section{La revolución en Francia}
\section{El concordato con Francia}
\section{La revolución en los demás países}
\section{Pío VII y Napoleón}
\chapter{Capítulo Segundo}
\section{La Iglesia y el Estado de Alemania}
\section{Despotismo del Estado}
\section{Los concordatos alemanes}
\section{Despertar de la conciencia católica en Alemania}
\section{Impugnación del regalismo y el liberalismo}
\chapter{Capítulo Tercero: La Iglesa Católica en los demás estados}
\section{El Papado e Italia}
\section{Austria-Hungría}
\section{La Iglesia en Suiza}
\section{Francia y los Países Bajos}
\section{España y Portugal. La América Latina}
\section{La Iglesia en la Gran Bretaña y América del Norte}
\section{La Iglesia en Rusia y Escandinavia}
\chapter{Capítulo Cuarto: Vida interior y extensión de la Iglesia}
\section{Desarrollo doctrinal. Sectas. Direccionse teológicas}
\section{La ciencia Eclesiástica}
\section{La vida monástica}
\section{El arte cristiano}
\section{Disciplina, culto, vida cristiana}
\section{Extensión separados de la Iglesia}
\chapter{Capítulo Quinto: Cristianos separados de la Iglesia}
\section{El Protestantismo}
\section{Las iglesias griega y rusa}
\section{Conclusión}
\printbibliography[heading=bibintoc]
\end{document}